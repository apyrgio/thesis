\chapter{Archipelago}\label{ch:archip}

\section{Overview}

Archipelago is the Volume Service of the Synnefo cloud software. It is 
responsible for creating Copy-on-Write, snapshottable volumes for VMs.  
Archipelago can be considered as a storage layer (see Figure 
\ref{fig:archipelago_overview_a.pdf}) that is positioned between the VM's block 
device, and the underlying storage.

\diagram{Archipelago overview}{archipelago_overview_a.pdf}

Archipelago has the following objectives:

\begin{itemize}
	\item Thinly provision volumes to VMs with zero data movement.
	\item Snapshot VM volumes and use them as system images with, again, zero 
		data movement.
	\item Allow VM migrations between Archipelago nodes with no restrictions.
	\item Be agnostic to the actual storage backend used.
\end{itemize}

\section{Architecture}

Archipelago has a modular architecture, which allows it to categorize its 
operations and assign them to distinct components. The IPC between these 
components, which are called \textbf{peers} in the Archipelago dialect, is 
facilitated by XSEG.

XSEG can be considered as the kernel of Archipelago and is a custom mechanism 
that:
\begin{inparaenum}[i)]
\item defines a common communication protocol for all peers, regardless of 
	their type (userspace/kernelspace, singlethreaded/multithreaded) and
\item builds a shared memory segment, where peers can share data using 
	zero-copy techniques.
\end{inparaenum}
The above are provided to the peers by the \texttt{libxseg} library.

In Figure \ref{fig:new_sxima_numbered.pdf}, we present the architecture of 
Archipelago.  Moreover, we show the Archipelago peers, the communication 
channels between them and we briefly explain the operations that they are 
responsible for.

\diagram{Archipelago components}{new_sxima_numbered.pdf}

\begin{description}
	\item[XSEG Block Device (xsegbd)] \hfill \\
		xsegbd is a kernel module that exposes a VM's volume as a block 
		device. For each VM, Archipelago registers an xsegbd block 
		device. This block device provides the entry point for the 
		requests that enter the Archipelago layer.
	\item[VoLuMe Composer Daemon (vlmcd)] \hfill \\
		vlmcd accepts requests from the various xsegbd devices and 
		translates them to object requests, with the help of mapperd.
	\item[Mapper Daemon (mapperd)] \hfill \\
		mapperd is responsible for the mapping of volumes to objects.  
		This means that it must tackle a broad set of tasks such as 
		knowing the objects that a volume consists of, cloning and 
		snapshotting volumes and creating new ones.
	\item[File Blocker Daemon (blockerd)] \hfill \\
		blockerd is not a specific entity but a family of drivers, each 
		of which is written for a specific storage type. File blockers 
		have a single purpose, to read/write objects from/to the 
		storage. Currently, there are file blockers for NFS and the 
		RADOS object storage.
\end{description}

\section{Requests}

We have already explained that Archipelago manipulates the requests that reach 
to the VM's block device. There are two main reasons behind this approach:

\begin{enumerate}
	\item Since Archipelago has its own CoW policy, the only way to enforce 
		it is to translate these requests to object requests and then 
		consult its mappings to find the actual data.
	\item In order to be able to use different storage backends, the 
		requests must be translated to native requests for the 
		respective backend.
\end{enumerate}

The request flow can be seen in Figure \ref{fig:new_sxima_numbered.pdf}. We 
will explain what happens in every step of the process, with the same order as 
they are presented in Figure \ref{fig:new_sxima_numbered.pdf}:

\begin{enumerate}
	\item The VM issues an I/O request, which is essentially a block 
		request. The hypervisor forwards this request to xsegbd, the 
		block device that is attached to every VM.
	\item The xsegbd's purpose is to bring this requests to the user level, 
		by copying its data to the shared memory segment and sending an 
		XSEG request to vlmcd
	\item Once the vlmcd receives the request, it translates it to an 
		object request. The object name can be constructed from the
		volume name and the dividend of the request offset divided by 
		the object size. Once vlmcd knows the target object, it asks 
		the mapper if this object is unique or a copy of another 
		object.
	\item The mapperd generally has the mappings of an object cached in its 
		memory, so it can respond quickly. In the rare event that they 
		are not cached, it queries a special purpose blocker to 
		retrieve them.
	\item At this point, mapperd has two courses of action:
		\begin{enumerate}[i)]
			\item it can either respond to vlmcd the correct object 
				that it must read from or write to (3) or
			\item perform a CoW operation (5)->(4)->(3)
		\end{enumerate}

		The CoW operation will be performed if the VM has asked to 
		write on an object which is a copy of another object. In this 
		case, mapperd will send a copy request to the data blocker and 
		will create a new object with the same data (5). Then, mapperd 
		will update its mapping and store it using the blocker for 
		mappings (4). Once all of the above have finished, mapperd will 
		respond to vlmcd the name of the copied object (3).
	\item At this point, vlmcd can send the request for the correct object 
		to the data blocker.
	\item Finally, the blocker will read/write the requested data to the 
		storage type that it has been written for.
\end{enumerate}

Once the storage backend completes the request, the data blocker notifies the 
vlmcd, which in turn notifies the xsebd, which finally notifies the VM.

\subsection{Request polling}\label{sec:arch-poll}

There is a part that has been omitted in the previous section and that is how 
peers send and receive requests. This is a rather intricate aspect of the 
Archipelago IPC, but we will explain it here since it is needed to understand 
mainly the synapsed implementation.

To begin with, one of the features of the shared segment is that it provides a 
number of \textbf{custom ports}. These ports are similar to network ports, in 
the sense that peers bind them and can listen for requests through them. Each 
port has essentially two queues:
\begin{inparaenum}[i)]
\item a \textbf{request queue}, where new requests are accepted, and
\item a \textbf{reply queue}, where replies to requests are received.
\end{inparaenum}

Thus, when a peer wants to send a request to another peer, it merely needs to 
enqueue that request to the request queue of the peer's port. Likewise, when a 
peer responds to a request that was sent by another peer, it will enqueue that 
request in the reply queue of that peer.

Moreover, peers are notified about new requests via signals. When a peer 
receives a signal, it wakes up and then accepts the request. However, context 
switching is expensive, so once the peer has served a request, it will not 
sleep immediately. Instead, it will iterate its ports for new requests for a 
small number of cycles and if no request is received until then, it will sleep.

\section{RADOS and sosd}\label{sec:rados-archip}

As we have mentioned above, Archipelago is not restricted to a storage backend 
and can work with any backend for which a blocker can be created. There is one 
backend however that provides many important features that are used to make 
Archipelago more scalable and reliable.


This backend is RADOS\cite{rados}, which is the object store component of the 
Ceph \footnote{ceph.com} system. Ceph is a free distributed object store and 
file system that has been created by Sage Weil for his doctoral dissertation
\cite{weil-thesis} and has been supported by his company, Inktank, ever since.

The architecture of Ceph can be seen in Figure \ref{fig:ceph.png}.
\footnote{Picture retrieved from the official website on September 24, 2013.  
	All rights go to the respective owners.}

\diagram{Ceph architecture}{ceph.png}

Ceph utilizes RADOS to achieve the following:

\begin{itemize}
	\item \textit{Replication}, which means that there can be many copies 
		of the same object so that the object is always accessible, 
		even when a node experiences a failure.
	\item \textit{Fault tolerance}, which is achieved by not having a 
		single point of failure. Instead, RADOS uses elected servers 
		called \textbf{monitors}, each of which have mappings of the 
		storage nodes where the objects and their replicas are stored.  
	\item \textit{Self-management}, which is possible since monitors know 
		at any time the status of the storage nodes and, for example, 
		can command to create new object replicas if a node experiences 
		a failure.
	\item \textit{Scalability}, which is aided by the fact that there is no 
		point of failure, which means that adding new nodes 
		theoretically does not add any communication overhead.
\end{itemize}

In a nutshell, RADOS consists of the following components:

\begin{itemize}
	\item \textit{object store daemons}, which are userspace processes that run 
		in the storage backend and are responsible for storing the data.
	\item \textit{monitor daemons}, which are monitoring userspace processes 
		that run in an odd number of servers that form a Paxos part-time 
		parliament\cite{Paxos}. Their main responsibility is holding and 
		reliably updating the mapping of objects to object store daemons, as 
		well as self-healing when an object store daemon or monitor daemon has 
		crashed.
\end{itemize}

The mapping of objects to object store daemons is done indirectly with the help 
of the \textit{CRUSH map} and \textit{placement groups (pgs)}.  Generally 
speaking, placement groups can be considered as logical buckets where more than 
one values can be assigned to.

RADOS uses placement groups in the following manner: On initialization, it is 
configured to have a fixed number of placement groups and a range of values 
assigned to each of them. When it is asked to find an object, the object name 
is hashed and its hash value is used to find in which placement group it 
belongs.  The relationship between placement groups and object store daemons is 
stored in CRUSH maps that each monitor daemon holds. This way, the objects are 
pseudorandomly distributed to the storage backend, which in turn implicitly 
guarantees load balancing. 

There are various entry points for RADOS, as is evident from Figure 
\ref{fig:ceph.png}. Archipelago uses librados for I/O requests and more 
specifically, a blocker called sosd\cite{sosd} has been created by Filippos 
Giannakos to facilitate the communication between RADOS and Archipelago.


\begin{comment}
\section{XSEG design}

The shared memory segment provides, besides a common IPC, the following to the 
peers that join it:

\begin{enumerate}
	\item A fast way to synchronize and/or have access to each other's data 
		using relative pointers (XSEG pointers) to the start of the 
		segment.
	\item A custom memory allocation mechanism that allocates data from the 
		segment, taking into account the memory fragmentation issues.
	\item Structures such as hash tables and stacks, that can be used and 
		accessed from different processes.
	\item An indication for the memory that is used by Archipelago.
\end{enumerate}

To achieve the above, XSEG utilizes several components that fall under the 
following main categories:

\begin{description}
	\item[Drivers] \hfill \\
		Drivers implement the common communication and memory 
		allocation protocol for every peer type (userspace/kernelspace, 
		single-threaded/multi-threaded).
	\item[Libraries] \hfill \\
		Libraries (or libxseg to be exact) are linked with peers and 
		allow them to use xtypes (see below) as well as useful generic 
		functions, such as logging.
	\item[Xtypes] \hfill \\
		Xtypes are data structures such as hash tables, queues, stacks 
		etc. whose data reside in the segment and can be used 
		concurrently by all peers. In fact, it is a rule in Archipelago 
		that if a structure must be accessed by more that one peers, it 
		must be designed as an XSEG type that will provide zero-copy 
		guarantees and integration with the segment.
	\item[Peers] \hfill \\
		Peers, which have been discussed above, are the components that 
		are responsible for accepting, processing and sending I/O 
		requests.
\end{description}

\subsection{Drivers}

\subsection{Libraries}

\subsection{Xtypes}\label{sec:arch-xtypes}

The rationale behind xtypes is:

\begin{itemize}
	\item Abstraction(?) layers: Creating inner abstractions layers for 
		software is not a new concept but it's very easy to miss, 
		especially when you start small and end up big.
		
		In a nutshell, when writing code for a new software (in our case 
		a peer for Archipelago but this can apply to most software that 
		surpass the 1000 LOC\footnote[1]
		{Lines Of Code}
		mark) it is wrong practice to create from scratch a monolithic 
		implementation with indistinguishable parts. There is a main 
		reason for this:
		
		Monolithic implementations usually derive from lack of code 
		architecture and planning. Although it is feasible for a 
		programmer to create fully-functional code that meets the 
		necessary requirements, albeit with a lot more effort and 
		concentration, this approach will backfire when the programmer 
		needs to add new features. Since there is no explicit code 
		architecture and the fragile inner correlations are between 
		lines of code and not separate entities, stored precariously in 
		the developer's mind, the result will eventually be constant 
		code refactorization.
		
		One might think that new features happen once in a while in the 
		development cycle but that would be wrong.  This happens more 
		often than you might think and is actually the common case in 
		iteration and test-driven development.

		The right practice instead is to...
	\item Re-usability:...
	\item User-space / Kernel-space agnosticity: (I doubt that such a word 
		even exists...)
\end{itemize}	

\subsection{Peers}\label{sec:arch-peer}

Peers are Archipelago components that are responsible for accepting, processing 
and sending of the I/O requests. They are essential for the modular nature of 
Archipelago since each of them can be considered as a separate entity. They do 
their own logging, signal handling and processing.

The main Archipelago peers can be seen in Figure ?. As we can see from this 
figure, peers are processes that are attached to an XSEG segment. In the 
previous chaptr, we have mentioned that XSEG segments facilitate the IPC between 
different Archipelago components by offering a shared space where process can 
read and write to very fast. This however barely scratches the surface of IPC in 
Archipelago. In the following section, we will discuss more in-length the 
details behind Archipelago IPC

\section{Archipelago IPC}\label{sec:arch-ipc}

An important part of the shared segment that was intentionally skipped in the 
previous section is its ports.

When XSEG is being initialized, it reserves space in the segment for a number 
of custom ports.  These ports are similar to network ports, in the sense that 
peers bind them and can listen for requests through them. Each port has 
essentially two queues:
\begin{inparaenum}[i)]
\item a request queue, where new requests are accepted, and
\item a reply queue, where replies to requests are received.
\end{inparaenum}
XSEG requests do not differ drastically from common requests and typically have 
a target, offset, size and a buffer to store the data. Targets can either be 
objects or volumes, depending on the peer context.

Another important aspect of the Archipelago IPC, is the mechanism with which 
peers poll for new requests, which is seen in the following section.

\subsection{Polling}
\label{sec:arch-poll}

The polling process \ref{sec:arch-poll}is the following:

\begin{enumerate}
	\item On peer initialization, the SIGIO signal is blocked and the PID 
		of the peer is masked from other peers.
	\item On the request loop, the peer checks its ports, without blocking
		and for a limited amount of cycles.
	\item On the last cycle, it unmasks its PID and check its ports one 
		last time.
	\item If there are no enqueued requests, it uses sigtimedwait to sleep 
		for 10s. sigtimedwait is a function that allows to wait 
		synchronously for pending signals, which is generally faster 
		than invoking a signal handler.
	\item If the 10s expire, the peer masks again its PID and goes to step 
		2.
\end{enumerate}

To understand what requests are in XSEG context, we present some parts of the 
xseg\_request structure:

\begin{description}
	\item[source port:] The port from
\end{description}

in particular have basically the following structure:

For instance, queues are  allocated in the segment, both queues and request are 
custom-built entities for XSEG
The interesting part of this process is


Fist of all, we must clarify that in Archipelago, IPC is done strictly between 
peers in the \textbf{same} memory segment. The reason is that we have crafted 
our own methods for IPC and the processes that need it must attain to a certain 
architecture, which is the peer architecture.

The entrance point for IPC is the peer port. When a peer is registered in the 
segment, it attaches itself to a port range. Peer ports are completely different 
to common ports (which are these ports?). When a peer wants to send a request to 
another peer, it must first "get" the registered port on the segment. The xseg 
port is a structure that holds the necessary information as to where to send the 
request. Every port has three different queues; reply, request, free queue.

Request queues are typically a stack that can be addressed from different peers 
in the same segment. For this reason, they are designed as xtypes. For speed 
resons, they are pre-allocated to a certain length and re-allocated on-line, if 
there is need

Also, ports are designed to be considered as paths. That is, when a request is 
sent from one port to another...

\section{Request flow example}

We have bench xseg which works like so:

\begin{enumerate}
	\item Get request
	\item Prepare request
	\item Create chunk
	\item Allocate peer request
	\item Set request (xhash)
	\item Submit request
\end{enumerate}

\subsection{Get request}\label{sec:get-req-archip}

Explain here about xq or in xtypes?
\end{comment}

