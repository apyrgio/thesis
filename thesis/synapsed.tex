\chapter{The Synapsed peer}

On the previous sections, we have evaluated the design of cached and we have 
seen that is heavily bounded by CPU. The implications of this, however,
are bigger, if we consider that all the Archipelago is running in the host 
machine, whose CPU's are already oversubscribed. This means that cached must 
compete for CPU time against the running VMs, essentialy defeating the QoS 
purpose it serves.

On the other hand, on the RADOS nodes, the CPUs are not used to the full 
potential. Thus, it would be interesting to check how well cached (and 
Archipelago in general) would behave, if the VM's data where send initially 
over network and then handled by the vlmc. So what we mean is to turn the 
current situation from how it is in Figure ? to the following:

\fixme add figure for network

To this end, we have created a network peer called synapsed (from 
\textbf{synapse d}aemon) as a proof-of-concept, that should be able to receive 
any type of requests and send them through the network to another Archipelago 
segment. We must note that this peer has not been created with performance in 
mind but its main aim is to provide the functionality needed for our purposes.  
As a consequence, we haven't used tools such as ZeroMQ or libevent that would 
boost the performance of the implementation.

More specifically, on Section...

\section{Design of synapsed}

Given that currently Archipelago is not network-aware, peers from one segment 
cannot know the ports of peers of another segment. Moreover, they cannot send a 
request to the synapsed peer and simultaneously target a peer in another 
segment. So, we are faced with the problem of what will be the limitations of 
synapsed.

We propose the following solution to this problem: Each of those two segments 
must have at least one synapsed peer. So, when one synapsed accepts an XSEG 
request, it will translate it to a network request, send it to the synapsed 
peer of the other segment which will finally translate it back to an XSEG 
request.  Moreover, each synapsed peer must be attached to a peer of its 
segment, which will serve as the request target when synapsed accepts request.  

This means that synapsed does not actually connect two remote segments
but bridge two remote peers over network.

The way synapsed handles requests can be seen in Figure ?

\fixme add figure

The main difficulty is to make synapsed listen from its port and its request 
queue simultaneously. We have tackled this problem in Section ?

Moreover, synapsed sends requests using bare TCP sockers, without using. This 
means that it handles the marshaling of requests, send/recv errors as well as 
polling by itself.

Finally, synapsed has been created as a single-threaded peer.

\section{Implementation of synapsed}

In the following sectionσ we will

\section{Synapsed initialization}

Synapsed is a much simpler peer than cached and requires only the following 
arguments.

\begin{description}
	\item[-hp] The port where it will listen for requests.
	\item[-ra] The remote address of the segment. \textbf{NOTE:} it can 
		point to the same segment, which means that synapsed provides a 
		generic way to bridge two peers who reside in the same host but 
		in a different segment.
	\item[-rp] The remote port of the other synapsed.
	\item[-txp] The XSEG port of the peer which synapsed will attach on
\end{description}

Having the above arguments, synapsed can create a socket and bing its port it.  

\section{Request polling}

Synapsed typically uses the common peer loop and Archipelago IPC methods to 
check its XSEG ports for new requests. However, it simultaneously needs to 
listen for new requests on its socket. Yet, there is no way for a process to 
sleep both on its socket and while waiting for a signal.

The solution seems obvious; synapsed should instead sleep while polling for 
requests in its socket and, if a new request arrives at its XSEG port, its 
polling will be interrupted. This is certainly the way to go, but there is one
detail we must take into account: Sleeping like so is unsafe because a signal 
may arrive right before we sleep and we will not take notice of it.

Let's elaborate on it a bit: The Archipelago IPC needs to block the SIGIO 
signal, so that sigtimedwait() can check the pending signals on sigset and 
wakeup when it receives one. In the poll case though, if we sleep with the 
SIGIO signal blocked, we will not wake up. On the other hand, if we unblocked 
the SIGIO signal, we would also have problems.

Consider the case where we receive that signal before we go to sleep. In order 
to be notified that a signal has been sent so as not to sleep, we would need to 
utilize a signal handler that would increment a global variable which would be 
checked right before we go to sleep. This approach is not only racy (what 
happens if SIGIO is received after this global variable is checked) but it is 
also very slow.

For this reason, we use the ppoll alternative, which accepts a signal mask as 
an additional argument. Ppoll provides the guarantee that it will use this 
signal mask during This way, ppoll can unblock the SIGIO signals while it 
sleeps and wake updhis signal mask is used first to block signals, then handle 
any signals that have managed to come in the meantime.  Ifsignals that can wake 
up the process.  ppoll
