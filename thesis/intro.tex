\chapter{Introduction}\label{ch:intro}

The racing track was clear and the computer hardware companies were tentatively 
positioning themselves on the starting blocks, in the April of 1965. Gordon E.  
Moore was holding the starter pistol, a paper where he famously stated that:

\begin{quotation}
	"The complexity for minimum component costs has increased at a rate of 
	roughly a factor of two per year. Certainly over the short term this rate 
	can be expected to continue, if not to increase. Over the longer term, the 
	rate of increase is a bit more uncertain, although there is no reason to 
	believe it will not remain nearly constant for at least 10 years. That 
	means by 1975, the number of components per integrated circuit for minimum 
	cost will be 65,000. I believe that such a large circuit can be built on a 
	single wafer."\cite{Moore}
\end{quotation}

Until that year, the integrated circuits (or microchips, as they are commonly 
called) were used prominently in embedded systems. Around the time the paper 
was published however, they were also starting to being adopted by computer 
manufacturers. The replacement of vacuum tubes with microchips marked the 
passage of the mainframe computer to the minicomputer and on to the personal 
computer that we know today.

Whether Moore's statement was a very accurate prediction of the future of 
microchips or a self-fulfilling prophecy that hardware companies used to market 
their products, is unsure. What is sure though is that his statement propelled 
the development of technology in general, since what became later on as a 
\textit{"law"} has been applied to numerous other technologies, unrelated to 
microchips such as the pixels of a camera or network capacity.  Wherever 
Moore's law was applicable, the industry would excel itself to adhere as much 
as possible to the projected growth, be it in microchip density, pixel density 
or performance.

Storage components, e.g. Hard Disk Drives (HDDs), Random Access Memory (RAM) 
and Flash Memory, had also entered the race and their capacity has been 
increasing ever since. In the performance track however, hard disks seem old 
and gasping to catch up the much faster RAM and CPU caches. For decades now, 
their sub-par performance has been the bottleneck of every IO-intensive 
application and the headache of storage designers \cite{nvm}.

As exaggerated as this might seem, their limitations have shaped the way 
storage is built; hardware solutions such as the RAID technology, 
battery-backed volatile memory in large servers and software solutions such as 
the Linux's page cache, memcached, bcache, are all notable examples which show 
that there is a tremendous effort that is being invested in sidestepping hard 
disks and finding alternative methods to store data.

The HDD's industry answer to this is the continuous drop of their prices.  In 
2011, HDDs reached their all-time low price of \$0.053/GB \cite{hdd-price}.  
Moreover, the emerging movement of greener data centers has benefited hard 
disks, since their lower energy costs than RAM is attractive to enterprises.  
Yet, for how long can the HDD industry keep lowering their costs to mitigate 
their lack of performance?

The answer came very fast and unfortunately in a tragic way. The end of July of 
2011 marked the beginning of a 6-month turmoil for Thailand, with a flood that 
was described as "the worst flooding yet in terms of the amount of water and 
people affected" \cite{flood}. The hard disk industry also suffered a huge hit 
due to the fact that 25\% percent of the global hard disk production was from 
factories in Thailand, that were largely affected by the flood.

The result was an overnight 40\% percent increase of hard disk prices. The 
reasons behind this increase were in one part to compensate for the flood 
damages and in another part to seize the opportunity to increase the profit 
margins of the two biggest producers, Western Digital and Seagate, from 6\% and 
3\% to 16\% and 37\% respectively \cite{rosenthal12-unesco}.

The timing could not have been worse for the HDD industry. The price increase 
led indirectly to the introduction of the more expensive but faster SSDs to the 
enterprise world. Their vast price drop \cite{ssd-price,ssddrop} in the last 
few years has made them viable candidates at least for peripheral storage tasks 
such as journaling and caching, and has led experts to consider them as the 
successors of HDDs.

On the other hand, HDDs are unable to retort performance-wise and can only 
marginally improve their performance.  As their rotational speed approaches the 
speed of sound, their production will be rendered at best difficult, and their 
heat generation, power consumption and lack of long-term reliability will make 
their adoption prohibitive \cite{hddtrends,speed-of-sound}.

Our prediction is that in some decades from now, when the dust will settle, the 
data centers will probably migrate from HDDs to SSDs. Till date however, the 
storage landscape is baffled with uncertainty as the tug-of-war between SSDs 
and HDDs is still at large. Moreover, besides SSDs, there are various other 
flash memory types, such as the IOdrive of Fusion IO, that are being utilized 
in performance-intensive environments, albeit for higher prices. Besides 
hardware solutions, there have also been developed various caching and 
buffering techniques to increase the performance of databases and storage in 
general.

To sum up, the current storage landscape provides the storage designer with 
various choices, each of which has its own merits and disadvantages. It is up 
to the storage designer to weigh these choices and implement the solution that 
fits the most to the profile of storage he/she builds and the budget of the 
storage service.

\section{Thesis motivation and background}

The motivation behind this thesis emerged from concerns about the storage 
performance of the Synnefo
\footnote{www.synnefo.org/} cloud software, which powers the \okeanos
\footnote{https://okeanos.grnet.gr/} public cloud service \cite{okeanos}. We 
will briefly explain what \okeanos and Synnefo are in the following paragraphs.

\okeanos is an IaaS (Infrastructure as a Service) that provides Virtual 
Machines, Virtual Networks and Storage services to the Greek Academic and 
Research community. It is an open-source service that has been running in 
production servers since 2011 by GRNET S.A.
\footnote{Greek Research and Technology Network, https://www.grnet.gr/}

Synnefo\cite{synnefo} is a cloud software stack, also created by GRNET S.A., 
that implements the following services which are used by \okeanos:

\begin{itemize}
	\item \textit{Compute Service}, which is the service that enables the 
		creation and management of Virtual Machines.
	\item \textit{Network Service}, which is the service that provides network 
		management, creation and transparent support of various network 
		configurations.
	\item \textit{Storage Service}, which is the service responsible for 
		provisioning the VM volumes and storing user data.
	\item \textit{Image Service}, which is the service that handles the 
		customization and the deployment of OS images.
	\item \textit{Identity Service}, which is the service that is responsible 
		for user authentication and management, as well as for managing the 
		various quota and projects of the users.
\end{itemize}

This thesis will deal exclusively with the Volume Service of Synnefo and more 
specifically with the part that handles the VMs' data and volumes, which is 
called Archipelago and is presented in Chapter \ref{ch:archip}.

As we have mentioned at the start of this section, the motivation behind this 
thesis was the unsatisfactory performance of Archipelago. More specifically, we 
measured the performance of VMs with and without caching enabled on the host 
and the results showed that host-side caching improved greatly the Archipelago 
performance.

Since Archipelago had no caching mechanisms and data were committed directly to 
a replicated object storage (see more in Section \ref{sec:rados-archip}) that 
was using hard disks, we concluded that this was the main reason behind the 
performance issues of Archipelago and decided to implement a caching system 
that would use a much faster storage medium.

As a result, this thesis presents the implementation of a caching system and 
documents all the design decisions and the thinking process that has led to 
this implementation.

\section{Thesis structure}

The thesis is organized as follows:

\begin{description}
\item[Chapter~\ref{ch:theory}:] \hfill \\
	We provide the necessary theoretical background for the concepts and 
	entities that are being discussed throughout the thesis.
\item[Chapter~\ref{ch:archip}:] \hfill \\
	We present the architecture of Archipelago and explain how Archipelago 
	handles I/O requests. Moreover, we provide information about RADOS, one 
	of the storage backends of Archipelago, as well as sosd, an Archipelago 
	component that has been created to communicate with RADOS and has been 
	the subject of a previous CSLab thesis \cite{sosd}.
\item[Chapter~\ref{ch:triad}:] \hfill \\
	We explain three of the most discussed concepts in the cloud world; 
	scalability, tiering and caching, and depict how they can be used in a 
	real-life scenario. Then, we present the most popular solutions for 
	increasing the storage performance of an application, such as bcache, 
	memcached, flashcache. Finally, we weigh their pros and cons explain 
	why they are inadequate for our purposes.
\item[Chapter~\ref{ch:cached-design}:] \hfill \\
	We explain the design of cached and the building blocks that is 
	consisted of (xcache, xworkq, xwaitq). Moreover, we illustrate how 
	cached handles I/O requests.
\item[Chapter~\ref{ch:cached-implementation}:] \hfill \\
	We present the cached implementation, such as the structures and 
	methods that we have used, in the form of code snippets. Furthermore, 
	we accompany them with the necessary commentary.
\item[Chapter~\ref{ch:cached-evaluation}:] \hfill \\
	In this chapter, we benchmark the sosd and cached peers under various 
	scenarios and evaluate their performance. Also, we present the 
	methodology behind our benchmarks and the specifications of the test 
	beds.
\item[Chapter~\ref{ch:synapsed}:] \hfill \\
	We present synapsed, a complimentary component to cached, whose purpose 
	is to transfer Archipelago requests over the network and allow cached 
	to run in other nodes than the host.
\item[Chapter~\ref{ch:future}:] \hfill \\
	We provide some concluding remarks about our thesis and assess in what 
	extend has it managed to achieve the goals that where set.  Also, we 
	discuss our plans for future improvements and deeper integration with 
	Archipelago.
\end{description}
