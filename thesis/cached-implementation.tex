\chapter{Implementation of cached}\label{ch:cached-implementation}

In the previous chapter, we presented a design overview for cached and its 
components. In this chapter we will how the above design has been implemented, 
the structures and functions that have been created.

More specifically, sections ? - ? provide...

\section{Implementation of xcache}

%Enter C code here

\subsection{Concurrency control}

\paragraph{Reference counting}

The refcount model in xcache should be familiar to most people:

% Turn this to figure
\begin{itemize}
	\item When an entry is inserted in cache, the cache holds a reference 
		for it (ref = 1).
	\item Whenever a new lookup for this cache entry succeeds, the reference 
		is increased by 1 (ref++)
	\item When the request that has issued the lookup has finished with an 
		entry, the reference is decreased by 1. (ref--)
	\item When a cache entry is evicted by cache, the its ref is decreased 
		by 1. (ref--)
\end{itemize}

Some common refcount cases are:

\begin{itemize}
	\item active entry with pending jobs (ref > 1)
	\item active entry with no pending jobs (ref = 1)
	\item evicted entry with pending jobs (ref > 0)
	\item evicted entry with no pending jobs (ref = 0)
\end{itemize}

and, as always, the entry is freed only when its ref = 0.

