\chapter{Archipelago}\label{ch:theory}

In this chapter we introduce the all-mighty Archipelago. Pray ye of little 
faith, for the savior has come, and it has come buckets (cached buckets)

\section{Overview}

Archipelago is the Volume Service of the Synnefo cloud software. It is 
responsible for creating Copy-on-Write, snapshottable volumes for VMs.  
Archipelago can be considered as a storage layer (see Figure 
\ref{fig:archipelago_overview_a.pdf}) that is positioned between the VM's block 
device, and the underlying storage.

\diagram{Archipelago overview}{archipelago_overview_a.pdf}

Archipelago has the following objectives:

\begin{itemize}
	\item Thinly provision volumes to VMs with zero data movement.
	\item Snapshot VM volumes and use them as system images with, again, zero 
		data movement.
	\item Allow VM migrations between Archipelago nodes with no restrictions.
	\item Be agnostic to the actual storage used.
\end{itemize}
	
In Figure \ref{fig:new_sxima.pdf}, we present the components of Archipelago.  
These components, which are called peers in the Archipelago dialect, and their 
functionalities are briefly explained below:

\diagram{Archipelago components}{new_sxima.pdf}

\begin{description}
	\item[XSEG Block Device (xsegbd)] \hfill \\
		xsegbd is a kernel module that exposes a block device.  This block 
		device is the VM's volume, as far as Ganeti and KVM are concerned, and 
		is the entry point to the Archipelago layer.
	\item[VoLuMe Composer Daemon (vlmcd)] \hfill \\
		vlmcd accepts requests from the various xsegbd devices and translates 
		them to object requests.
	\item[Mapper Daemon (mapperd)] \hfill \\
		mapperd is responsible for the mapping of volumes to objects. This 
		means that it must tackle a broad set of tasks such as knowing the 
		objects that a volume consists of, cloning and snapshotting volumes and 
		creating new ones
	\item[File Blocker Daemon (blockerd)] \hfill \\
		blockerd is not a specific entity but a family of drivers, each of 
		which is written for a specific storage type. File blockers have a 
		single purpose, to read/write objects from/to the storage. Currently, 
		there are file blockers for NFS and the RADOS object storage.
\end{description}
	
Besides the Archipelago peers, there is an important component that facilitates 
the communication between them. This component is XSEG and is the core of 
Archipelago. It provides a shared memory segment upon which peers can share 
data using zero copy methods, as well as a custom communication method, common 
for all peers and independent from their domain (userspace or kernelspace).

\section{XSEG design}

The shared memory segment provides, besides a common IPC, the following to the 
peers that join it:

\begin{enumerate}
	\item A fast way to synchronize and/or have access to each other's data 
		using relative pointers (XSEG pointers) to the start of the 
		segment.
	\item A custom memory allocation mechanism that allocates data from the 
		segment, taking into account the memory fragmentation issues.
	\item Structures such as hash tables and stacks, that can be used and 
		accessed from different processes.
	\item An indication for the memory that is used by Archipelago.
\end{enumerate}

To achieve the above, XSEG utilizes several components that fall under the 
following main categories:

\begin{description}
	\item[Drivers] \hfill \\
		Drivers implement the common communication and memory 
		allocation protocol for every peer type (userspace/kernelspace, 
		single-threaded/multi-threaded).
	\item[Libraries] \hfill \\
		Libraries (or libxseg to be exact) are linked with peers and 
		allow them to use xtypes (see below) as well as useful generic 
		functions, such as logging.
	\item[Xtypes] \hfill \\
		Xtypes are data structures such as hash tables, queues, stacks 
		etc. whose data reside in the segment and can be used 
		concurrently by all peers. In fact, it is a rule in Archipelago 
		that if a structure must be accessed by more that one peers, it 
		must be designed as an XSEG type that will provide zero-copy 
		guarantees and integration with the segment.
	\item[Peers] \hfill \\
		Peers, which have been discussed above, are the components that 
		are responsible for accepting, processing and sending I/O 
		requests.
\end{description}

\begin{comment}
\subsection{Drivers}

\subsection{Libraries}

\subsection{Xtypes}\label{sec:arch-xtypes}

The rationale behind xtypes is:

\begin{itemize}
	\item Abstraction(?) layers: Creating inner abstractions layers for 
		software is not a new concept but it's very easy to miss, 
		especially when you start small and end up big.
		
		In a nutshell, when writing code for a new software (in our case 
		a peer for Archipelago but this can apply to most software that 
		surpass the 1000 LOC\footnote[1]
		{Lines Of Code}
		mark) it is wrong practice to create from scratch a monolithic 
		implementation with indistinguishable parts. There is a main 
		reason for this:
		
		Monolithic implementations usually derive from lack of code 
		architecture and planning. Although it is feasible for a 
		programmer to create fully-functional code that meets the 
		necessary requirements, albeit with a lot more effort and 
		concentration, this approach will backfire when the programmer 
		needs to add new features. Since there is no explicit code 
		architecture and the fragile inner correlations are between 
		lines of code and not separate entities, stored precariously in 
		the developer's mind, the result will eventually be constant 
		code refactorization.
		
		One might think that new features happen once in a while in the 
		development cycle but that would be wrong.  This happens more 
		often than you might think and is actually the common case in 
		iteration and test-driven development.

		The right practice instead is to...
	\item Re-usability:...
	\item User-space / Kernel-space agnosticity: (I doubt that such a word 
		even exists...)
\end{itemize}	

\subsection{Peers}\label{sec:arch-peer}

Peers are Archipelago components that are responsible for accepting, processing 
and sending of the I/O requests. They are essential for the modular nature of 
Archipelago since each of them can be considered as a separate entity. They do 
their own logging, signal handling and processing.

The main Archipelago peers can be seen in Figure ?. As we can see from this 
figure, peers are processes that are attached to an XSEG segment. In the 
previous chaptr, we have mentioned that XSEG segments facilitate the IPC between 
different Archipelago components by offering a shared space where process can 
read and write to very fast. This however barely scratches the surface of IPC in 
Archipelago. In the following section, we will discuss more in-length the 
details behind Archipelago IPC

\end{comment}

\section{Archipelago IPC}\label{sec:arch-ipc}

An important part of the shared segment that was intentionally skipped in the 
previous section is its ports.

When XSEG is being initialized, it reserves space in the segment for a number 
of custom ports.  These ports are similar to network ports, in the sense that 
peers bind them and can listen for requests through them. Each port has 
essentially two queues:
\begin{inparaenum}[i)]
\item a request queue, where new requests are accepted, and
\item a reply queue, where replies to requests are received.
\end{inparaenum}
XSEG requests do not differ drastically from common requests and typically have 
a target, offset, size and a buffer to store the data. Targets can either be 
objects or volumes, depending on the peer context.

Another important aspect of the Archipelago IPC, is the mechanism with which 
peers poll for new requests, which is seen in the following section.

\subsection{Polling}

The polling process is the following:

\begin{enumerate}
	\item On peer initialization, the SIGIO signal is blocked and the PID 
		of the peer is masked from other peers.
	\item On the request loop, the peer checks its ports, without blocking
		and for a limited amount of cycles.
	\item On the last cycle, it unmasks its PID and check its ports one 
		last time.
	\item If there are no enqueued requests, it uses sigtimedwait to sleep 
		for 10s. sigtimedwait is a function that allows to wait 
		synchronously for pending signals, which is generally faster 
		than invoking a signal handler.
	\item If the 10s expire, the peer masks again its PID and goes to step 
		2.
\end{enumerate}

\begin{comment}
To understand what requests are in XSEG context, we present some parts of the 
xseg\_request structure:

\begin{description}
	\item[source port:] The port from
\end{description}

in particular have basically the following structure:

For instance, queues are  allocated in the segment, both queues and request are 
custom-built entities for XSEG
The interesting part of this process is


Fist of all, we must clarify that in Archipelago, IPC is done strictly between 
peers in the \textbf{same} memory segment. The reason is that we have crafted 
our own methods for IPC and the processes that need it must attain to a certain 
architecture, which is the peer architecture.

The entrance point for IPC is the peer port. When a peer is registered in the 
segment, it attaches itself to a port range. Peer ports are completely different 
to common ports (which are these ports?). When a peer wants to send a request to 
another peer, it must first "get" the registered port on the segment. The xseg 
port is a structure that holds the necessary information as to where to send the 
request. Every port has three different queues; reply, request, free queue.

Request queues are typically a stack that can be addressed from different peers 
in the same segment. For this reason, they are designed as xtypes. For speed 
resons, they are pre-allocated to a certain length and re-allocated on-line, if 
there is need

Also, ports are designed to be considered as paths. That is, when a request is 
sent from one port to another...

\section{Request flow example}

We have bench xseg which works like so:

\begin{enumerate}
	\item Get request
	\item Prepare request
	\item Create chunk
	\item Allocate peer request
	\item Set request (xhash)
	\item Submit request
\end{enumerate}

\subsection{Get request}\label{sec:get-req-archip}

Explain here about xq or in xtypes?
\end{comment}
