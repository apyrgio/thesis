\chapter{Introduction}\label{ch:intro}

%In the technology race of computer components
In the new-age race of hardware, where vendors are always trying to produce the 
next record-breaking unit, be it CPU, RAM or network cards, hard disks seem old 
and gasping to catch up. For decades now, their sub-par performance, compared 
to volatile memory, has been the bottleneck of every IO-intensive application 
and the headache of storage designers \cite{nvm}. As over-stretched as this may 
seem, it is a fact that it has shaped the way storage is built; from the common 
computer (Linux's page cache, Bcache) to large deployments (the memcached 
servers of Facebook and Twitter \fixme add paper), there is a tremendous effort 
from the research and corporate community that is being invested in 
sidestepping hard disks and finding alternative methods to store data.

The hard disks' industry answer to this is the continuous drop of their prices.  
In 2011 (\fixme be more specific), the HDDs reached their all-time low price of 
\$0.053/GB \cite{hdd-price}. Moreover, the emerging movement of greener data 
centers has benefited hard disks, since their low energy costs has been proved 
attractive to the enterprises. Yet, for how long can the HDD industry keep 
lowering the costs to mitigate their lack of performance?

The answer came very fast and unfortunately in a tragic way. The end of July of 
2011 marked the beginning of a 6-month turmoil for Thailand, with a flood that 
was described as "the worst flooding yet in terms of the amount of water and 
people affected" \cite{flood}. The hard disk industry also suffered a huge hit 
due to the fact that 25\% percent of the global hard disk production was from 
factories in Thailand, that were also affected by the flood.

The result was an overnight 40\% percent increase of hard disk prices. The 
reasons behind this increase were partly to compensate for the flood damages 
and partly to seize the opportunity to increase the profit margins of the two 
biggest producers, Western Digital and Seagate, from 6\% and 3\% to 16\% and 
37\% respectively \cite{rosenthal12-unesco}.

The timing could not have been worse for the HDD industry. This increase in 
price introduced SSDs who are now starting to be considered as a viable 
solution for peripheral storage tasks such as journaling and caching, due to 
their high performance and their persistence, which separates them from other 
volatile storage types.

On the other hand, the current situation is that HDDs can only marginally 
improve their performance. As their rotational speed approaches the speed of 
sound, their production will be rendered at best difficult, and their heat 
generation, power consumption and lack of long-term reliability will make their
adoption prohibitive \cite{hddtrends},\cite{speed-of-sound}.

The cloud world is largely affected by the future of storage mediums. Besides 
SSDs, there are various other non-volatile storage types that are trying to 
race for the momentum




\todo Explain in a few words what we are trying to do here and why.


\section{Thesis structure}

\fixme Fix the summary and references of these chapters.

% TODO: Make Chapter x click-able and anchored to the respective chapter

\begin{description}
\item[Chapter~\ref{ch:ch2}:]
We define what "cloud" means and mention some of the most notable examples.
Then, we give a brief overview of the synnefo implementation, its key
characteristics and why it can have a place in the current cloud world.
\item[Chapter~\ref{ch:archipelago}:]
We present the architecture of Archipelago and provide the necessary 
theoretical background (mmap, IPC) the reader needs to understand its basic 
concepts. Then, we thoroughly explain how Archipelago handles I/O requests.  
Finally, we mention what are the current storage mechanisms for Archipelago and 
evaluate their performance.
\item[Chapter~\ref{ch:tiering}:]
We explain why tiering is important and what is the state of tiered storage at
the moment (bcache, flashcache, memcached, ramcloud, couchbase).  Then, we
provide the related theoretical background for cached (hash-tables, LRUs).
Finally, we defend why we chose to roll out our own implementation.
\item[Chapter~\ref{ch:cached-design}:]
We explain the design of cached, the building blocks that is consisted of
(xcache, xworkq, xwaitq). Then, we give some examples that illustrate the 
operation under different scenarios
\item[Chapter~\ref{ch:cached-implementation}:]
We present the cached implementation, the structures that have been created and 
the functions that have been used.
\item[Chapter~\ref{ch:cached-evaluation}:]
We explain how cached was evaluated and present benchmark results.
\item[Chapter~\ref{ch:synapsed}:]
It connects brain parts. And its tale must be told.
\item[Chapter~\ref{ch:ch7}:]
We draw some concluding remarks and propose some future work.
\end{description}
