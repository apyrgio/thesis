\chapter{Necessary theoretical background}\label{ch:theory}

In this typically boring section, we will explain the basic system facilities 
and programming concepts that are used by our implementation and Archipelago in 
general. More specifically, in Section ?...

\subsection{Interprocess Communication - IPC}

Interprocess Communication is a concept that predates the SMP
\footnote{Symmetric multiprocessing}
systems that we all use nowadays and is a set of methods that an OS uses to 
allow processes to communicate with each other. Although most applications are 
independent from each other. %page 37

Below we can see some IPC methods:

\begin{enumerate}
	\item mmap()
	\item Semaphores
	\item Sockets
\end{enumerate}

\subsection{Multi-threaded programming}

% http://accu.org/index.php/journals/1634

Multi-threading programming is good and is bad and here are some challenges:

\begin{enumerate}
	\item Concurrency control
	\item Challenge 2
	\item Challenge 3
\end{enumerate}

\subsubsection{Concurrency control}

\paragraph{Locking}

% Taken from wikipedia

Three concepts for locking:

\begin{enumerate}
	\item Lock overhead
	\item Lock contention
	\item Deadlocking
\end{enumerate}

\section{Archipelago}

Archipelago consists of the following:

1. XSEG
2.
3.

\section{XSEG}

XSEG is the segment on which the IPC...

There are some XSEG stuff such as:

1. Drivers
2. Libraries
3. Xtypes
4. Peers

\subsection{Drivers}

\subsection{Libraries}

\subsection{Xtypes}\label{sec:arch-xtypes}

The rationale behind xtypes is:

\begin{itemize}
	\item Abstraction(?) layers: Creating inner abstractions layers for 
		software is not a new concept but it's very easy to miss, 
		especially when you start small and end up big.
		
		In a nutshell, when writing code for a new software (in our case 
		a peer for Archipelago but this can apply to most software that 
		surpass the 1000 LOC\footnote[1]
		{Lines Of Code}
		mark) it is wrong practice to create from scratch a monolithic 
		implementation with indistinguishable parts. There is a main 
		reason for this:
		
		Monolithic implementations usually derive from lack of code 
		architecture and planning. Although it is feasible for a 
		programmer to create fully-functional code that meets the 
		necessary requirements, albeit with a lot more effort and 
		concentration, this approach will backfire when the programmer 
		needs to add new features. Since there is no explicit code 
		architecture and the fragile inner correlations are between 
		lines of code and not separate entities, stored precariously in 
		the developer's mind, the result will eventually be constant 
		code refactorization.
		
		One might think that new features happen once in a while in the 
		development cycle but that would be wrong.  This happens more 
		often than you might think and is actually the common case in 
		iteration and test-driven development.

		The right practice instead is to...
	\item Re-usability:...
	\item User-space / Kernel-space agnosticity: (I doubt that such a word 
		even exists...)
\end{itemize}	

\subsection{Peers}\label{sec:arch-peer}

Peers are Archipelago components that are responsible for accepting, processing 
and sending of the I/O requests. They are essential for the modular nature of 
Archipelago since each of them can be considered as a separate entity. They do 
their own logging, signal handling and processing.

The main Archipelago peers can be seen in Figure ?. As we can see from this 
figure, peers are processes that are attached to an XSEG segment. In the 
previous chaptr, we have mentioned that XSEG segments facilitate the IPC between 
different Archipelago components by offering a shared space where process can 
read and write to very fast. This however barely scratches the surface of IPC in 
Archipelago. In the following section, we will discuss more in-length the 
details behind Archipelago IPC

\subsection{Archipelago IPC}\label{sec:arch-ipc}

Fist of all, we must clarify that in Archipelago, IPC is done strictly between 
peers in the \textbf{same} memory segment. The reason is that we have crafted 
our own methods for IPC and the processes that need it must attain to a certain 
architecture, which is the peer architecture.

The entrance point for IPC is the peer port. When a peer is registered in the 
segment, it attaches itself to a port range. Peer ports are completely different 
to common ports (which are these ports?). When a peer wants to send a request to 
another peer, it must first "get" the registered port on the segment. The xseg 
port is a structure that holds the necessary information as to where to send the 
request. Every port has three different queues; reply, request, free queue.

Request queues are typically a stack that can be addressed from different peers 
in the same segment. For this reason, they are designed as xtypes. For speed 
resons, they are pre-allocated to a certain length and re-allocated on-line, if 
there is need

Also, ports are designed to be considered as paths. That is, when a request is 
sent from one port to another...

\section{Request flow example}

We have bench xseg which works like so:

\begin{enumerate}
	\item Get request
	\item Prepare request
	\item Create chunk
	\item Allocate peer request
	\item Set request (xhash)
	\item Submit request
\end{enumerate}

\subsection{Get request}\label{sec:get-req-archip}

Explain here about xq or in xtypes?
