\begin{abstractgr}
	Η υπηρεσία αποθήκευσης δεδομένων μιας υποδομής υπολογιστικού νέφους 
	(cloud computing) έχει κατά κανόνα υψηλές απαιτήσεις σε επιδόσεις, 
	ειδικά όταν είναι επιφορτισμένη με την παροχή των εικονικών δίσκων για 
	τις Εικονικές Μηχανές (VMs) των χρηστών. Σε μεγάλες εγκαταστάσεις όμως, 
	παρατηρείται ότι η επίδοση των σκληρών δίσκων (hard disks) συνήθως δεν 
	επαρκεί για την εξασφάλιση γρήγορων ταχυτήτων για τις ανάγκες των 
	χρηστών. Αυτό οδηγεί τους προγραμματιστές στο να καταφεύγουν σε λύσεις 
	προσωρινής αποθήκευσης δεδομένων (caching) σε γρηγορότερα αποθηκευτικά 
	μέσα, για να βελτιώσουν τις επιδόσεις τους.
	
	Η παρούσα διπλωματική εργασία παρουσιάζει τον \textit{cached}, έναν 
	caching μηχανισμό ο οποίος δημιουργήθηκε για τη βελτίωση της επίδοσης 
	του Archipelago, ενός κατανεμημένου συστήματος αποθήκευσης σε 
	περιβάλλον υπολογιστικού νέφους (cloud computing environment).	
	Επιπροσθέτως, παρουσιάζεται το συμπληρωματικό δικτυακό εργαλείο 
	\textit{synapsed}, το οποίο ανοίγει το δρόμο για την αναβάθμιση του 
	\textit{cached} σε έναν πλήρως κατανεμημένο caching μηχανισμό.
	
	Τα αποτελέσματα από τις πρώτες μετρήσεις είναι αρκετά ενθαρρυντικά και 
	δείχνουν ότι η χρήση του \textit{cached} μπορεί να αυξήσει την επίδοση του 
	Archipelago μέχρι και 400\%.
	\begin{keywordsgr}
		cached, synapsed, synnefo, okeanos, rados, archipelago,  
		εικονική μηχανή, προσωρινή αποθηκευτική μνήμη, Ο(1) 
		πολυπλοκότητα, φυσική μνήμη, κατανεμημένο σύστημα, υπολογιστικό 
		νέφος, κρυφή μνήμη
	\end{keywordsgr}
\end{abstractgr}

\begin{abstracten}
	The performance of the storage service of a cloud infrastructure is very 
	critical, especially when it provides virtual volumes for the Virtual
	Machines of users. Underlying hard disks however, are usually inadequate in 
	providing fast performance for large deployments and the storage engineers 
	commonly have to resort in using various buffering and caching techniques.
	
	This diploma thesis presents \textit{cached}, a caching mechanism that 
	aims to improve the performance of Archipelago, a software-defined, 
	distributed storage layer for cloud computing environments.  
	Additionally, this diploma thesis introduces \textit{synapsed}, a 
	complementary network tool that paves the way for the creation of a 
	fully distributed cache.
	
	Early performance evaluations look very promising and show that 
	\textit{cached} can improve the current Archipelago performance up to 
	400\%.
	\begin{keywordsen}
		cached, synapsed, synnefo, okeanos, rados, archipelago, cloud, 
		virtual machine, volume service, cache, hash table,  O(1) 
		complexity, ram, distributed system, replication, 
		software-defined, RAM
	\end{keywordsen}
\end{abstracten}

\begin{acknowledgementsgr}
	Οι παρακάτω παράγραφοι ήθελα να είναι ένας επίλογος μιας προσπάθειας 
	που ξεκίνησε όταν μπήκα στη σχολή. Όσο όμως τις γράφω και τις σβήνω, 
	νιώθω πως υπάρχουν ακόμα κεφάλαια ανοικτά, που κάνουν τις αναφορές μου 
	σε φίλους και συγγενείς να φαίνονται πρώιμες. Ίσως σε κάποια άλλη 
	σελίδα άλλων ευχαριστιών μπορέσω να μιλήσω για αυτούς τους κοντινούς 
	μου ανθρώπους.

	Το κεφάλαιο που σίγουρα τέλειωσε όμως είναι αυτό της διπλωματικής. Γι' 
	αυτό λοιπόν, θα ήθελα να ευχαριστήσω κάποιους ανθρώπους που συνετέλεσαν 
	καταλυτικά στην ολοκλήρωσή της.
	
	Θα ήθελα αρχικά να ευχαριστήσω τον καθηγητή Νεκτάριο Κοζύρη, που μου 
	έδωσε την ευκαιρία να εργαστώ πάνω σε ένα πολύ ενδιαφέρον και φρέσκο 
	αντικείμενο, το cloud computing.

	Επίσης, θα ήθελα να ευχαριστήσω τον Υ.Δ. Φίλιππο Γιαννάκο για τις φορές 
	που με προσγείωνε όταν χρειαζόταν, που με διόρθωνε όπου είχα λάθος και 
	που μου	έδινε λύσεις όταν βρισκόμουν σε αδιέξοδο.

	Τέλος, θα ήθελα να ευχαριστήσω τον Δρα. Βαγγέλη Κούκη που, παρά τα 
	όποια κενά στις γνώσεις μου, με στήριξε στην επιλογή μου να ασχοληθώ με 
	κάτι τεχνικά δύσκολο και με βοήθησε να τα καλύψω στην πορεία.
\end{acknowledgementsgr}
