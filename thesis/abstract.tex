% DOCUMENT FORMAT ======================= -*- mode: LaTeX; coding: utf-8 -*- ===
\begin{abstractgr}
	\todo Φτιάξε την ελληνική περίληψη και τις λέξεις κλειδιά
\begin{keywordsgr}
  	Λέξη-κλειδί 1, λέξη-κλειδί 2, λέξη-κλειδί 3
\end{keywordsgr}
\end{abstractgr}

\begin{abstracten}
	The storage service of a cloud infrastracture is a very performance 
	critical part. Hard disks are commonly inadequate in providing the 
	performance that is needed for large deployments and the storage 
	engineers must always resort to other ways of sidestepping these 
	issues.
	
	Enter Synnefo, an open source cloud software that powers the \okeanos 
	cloud service, which is used by the Greek Academic and Research 
	Community. Towards the early stages of Synnefo, a nover distributed 
	storage layer called Archipelago has been created to tackle storage 
	issues by introducing Copy-on-Write logic to the storage of a VM's 
	volume. One of the storage backends that Archipelago uses is RADOS, 
	self healing, self-managing, distributed storage system. The 
	performance of RADOS has not been adequeate however, due to the fact 
	that is sits on hard disks.

	This thesis describes the third-party solutions that have been 
	evaluated to improve the performance of Archipelago and presents the 
	caching solution that we have decided to implement, cached.
	
	The design of cached is discussed in length and we present the 
	obstacles that we encountered as well as the implementation details of 
	cached. We have evaluated the performance of cached and seen that it 
	can provide up to 400\% speedup over our current performance.

	Finally, we present the design and implementation of sunapsed, a 
	component that paves the way to allow cached to be distributed.

	\begin{keywordsen}
		synnefo, okeanos, rados, archipelago, cached, cache, storage, 
		ram, replication, synapsed
	\end{keywordsen}
\end{abstracten}

\begin{acknowledgementsgr}
	\todo Thank everyone
\end{acknowledgementsgr}



