\section{Cached design}

\begin{frame}{Requirements}

	Design goals for cached:
	\begin{itemize}
		\item Create something close to the Archipelago logic
		\item Measure the best possible performance we can get
	\end{itemize}

	\note{Επιλέξαμε λοιπόν να δημιουργήσαμε τη δική μας λύση. Την ονομάσαμε 
		cached από το cache daemon}
	\note{Ορίσαμε τους εξής γενικούς στόχους:
		\begin{itemize}
			\item Δημιουργία ενός peer κοντά στη λογική του 
				Archipelago
			\item Η υλοποίηση να είναι όσο το δυνατόν πιο γρήγορη 
				για να δουμε αν μια Αρχιπελαγική λύση μας 
				βοηθάει
		\end{itemize}
	}
	\dspc
	Stricter requirements for cached:
	\begin{itemize}
		\item Nativity
		\item Pluggability
		\item In-memory
		\item Low indexing overhead
	\end{itemize}
	\note{Ακόμα, θέσαμε κάποιες πιο αυστηρές απαιτήσεις για την υλοποίηση 
		μας: 1)να είναι peer του Archipelago, 2) να μπορεί να 
		ενεργοποιείται και να απενεργοποιείται σε ένα σύστημα που 
		τρέχει,3)να χρησιμοποιεί τη RAM, 4)ο indexing μηχανισμός να 
		είναι γρήγορος}
\end{frame}

\begin{frame}{Cached design}
	\note{Εδώ βλέπουμε το design του cached. Ο cached μπαίνει ανάμεσα στον 
		vmlc και στον blocker και cach-άρει ότι άιτημα για αντικείμενα 
		πάει στο storage}
	\note{Οι εργασίες του cached χωρίζονται σε 5 κατηγορίες:}
	\note[item]{Στην διαχείριση των αιτημάτων από και προς vlmc, blocker}
	\note[item]{Στο indexing (εύρεση και καταχώρηση) των αντικειμένων}
	\note[item]{Στην ασύγχρονη ή κατά όρους εκτέλεση εργασιών}
	\note[item]{Στην ασφαλή μετάδοση των cachαρισμένων δεδομένων στο 
		storage}
	\note[item]{Καθώς επίσης και στην ασφαλή επεξεργασία των cachαρισμένων 
		δεδομένων}
	\note[item]{\click}
	\centering\includegraphics<1>[height=0.8\textheight]{images/cached-design2.pdf}
	\centering\includegraphics<2>[height=0.8\textheight]{images/cached-design-comp2.pdf}
	\note{item}{Και εδώ βλέπουμε τα διακριτά κομμάτια που υλοιποιούν τα 
		παραπάνω και τα οποία θα συζητήσουμε ευθύς αμέσως}
\end{frame}

\begin{frame}{Xcache design}
	\centering\includegraphics<1>[height=0.6\textheight]{images/xcache-design.pdf}
	\centering\includegraphics<2>[width=\textwidth]{images/xcache-entry.pdf}

	Xcache is responsible for: 1) entry indexing, 2) entry eviction, 3) 
	concurrency control

	\note[item]{Αυτό είναι το xcache, που είναι υπεύθυνο για την
		\begin{itemize}
			\item Eύρεση και καταχώριση των αντικειμένων
			\item Έξωση αντικειμένων από την cache με τη χρήση LRU
			\item Χειρισμό πολλαπλών threads
		\end{itemize}
	}
	\note[item]{Εχουμε ένα hash table <αυτό> που κρατάει τα ονόματα των 
		αντικειμένων. Ο αποθηκευτικός τους χώρος είναι preallocated και 
		είναι <αυτό>. Σε αυτό το χώρο, η αναφορά γίνεται με δείκτες. Ο 
		ελεύθερος χώρος είναι στη στοιβα αυτή. Τέλος, όταν ένα 
		αντικείμενο φύγει, μένει σε αυτό το hash table μεχρι να το 
		ξαναζητήσουν ή να το διωχτεί}
	\note[item]{\click}
	\note[item]{Κάθε item έχει ένα reference counter για να ξέρουμε πόσοι 
		το χρησιμοποιούν, όνομα, lru}
	\end{frame}

\begin{frame}{Xworkq design}
	\centering\includegraphics<1>[width=\textwidth]{images/xworkq-design.pdf}

	Xworkq is responsible for concurrency control
	\note[item]{xworkq υπεύθυνο για την ασφαλή επεξεργασία των δεδομένων 
		ενός αντικείμενου.}
	\note[item]{Το spinning είναι αργό, όλοι τοποθετούν μια δουλειά, ένας 
		την εκτελεί.}
	\end{frame}

\begin{frame}{Xwaitq design}
	\centering\includegraphics<1>[width=\textwidth]{images/xwaitq-design.pdf}

	Xwaitq is responsible for deferred execution
	\note[item]{xwaitq υπεύθυνο για την κατά συνθήκη εκτελεση εργασιών}
	\note[item]{Αν π.χ. μας τελειώσει ο χώρος, δεν μπορούμε να περιμένουμε 
		σύγχρονα. To thread μπορεί να τοποθετήσει μια δουλειά και μετά 
		να εκτελέσει κάτι άλλο}
\end{frame}

\begin{frame}{Bucket pool}
	When an object is indexed, it does not have immediate access to 4MB 
	size of data because:
	\begin{itemize}
		\item RAM is limited
		\item Leads to small number of entries
	\end{itemize}
	\note[item]{Το ότι κάνουμε index ένα object δε σημαίνει ότι κατ'ευθείαν 
		μπορούμε να γράψουμε σε αυτό}
	\note[item]{Δεν υπάρχει τόση RAM και ακόμα και ως αποτέλεσμα, θα 
		cachάραμε μικρό αριθμό από objects}
	Ideally, we want to:
	\begin{itemize}
		\item Decouple the objects from their data
		\item Cache unlimited objects but put a limit on their data
	\end{itemize}
	\note[item]{Ιδανικά θέλουμε να διαχωρίσουμε την καταχώρη/όνομα του 
		αντικειμένου από τα δεδομένα του. Δυνητικά θα μπορούμε να 
		καταχωρούμε πάρα πολλά αντικείμενα αλλά θα έχουμε μικρότερο 
		χώρο}

	Solution:
	\begin{itemize}
		\item Preallocated data space
		\item Every object request a bucket (typically 4KB)		
		\item When an object is evicted, its buckets are 
			reclaimed
	\end{itemize}
	\note[item]{Preallocated χώρος, όλοι παίρνουν indexes από αυτό (Θυμίζει 
		xcache}
\end{frame}

\begin{frame}{Other important cached tasks}

	Several other key-tasks are:
	\begin{itemize}
		\item Book-keeping
		\item Cache write policy
		\item Asynchronous task execution
		\item Data propagation
	\end{itemize}
	\note{Το cached είναι επίσης επιφορτισμένο και με άλλες δουλειές όπως:
		\begin{itemize}
			\item Κρατάει στατιστικά (πόσα entries είναι dirty, 
				πόσα buckets έχει κάνει allocate ένα entry
			\item Εφαρμόζει writeback/writethrough πολιτική
			\item Φρόντίζει ώστε οι εργασίες να μπορούν να γίνουν 
				ασύγχρονα
			\item Και φυσικά φροντίζει τα δεδομένα να γράφονται 
				σωστά στο storage
		\end{itemize}
	}
\end{frame}

\begin{frame}{Cached flow}
	\includegraphics<1>[height=0.8\textheight]{images/cached-design2.pdf}
	\includegraphics<2>[height=0.8\textheight]{images/cached-design-comp2.pdf}

	\note[item]{Εδώ παίζεις με τα slides}
	\note[item]{Θα παρουσιάσουμε πολύ γρήγορα τη ροή ενός αιτήματος στον 
		cached}
	\note[item]{Έρχεται request, το κάνουμε index, μπαίνουμε στη workq και 
		πειράζουμε τα δεδομένα του και ανάλογα το cache policy το 
		γράφουμε πίσω στον blocker αλλιώς τελειώσαμε}
	\note[item]{Optional σεναρια:
		\begin{itemize}
			\item Αν γίνει ένα eviction, πρέπει να γράψουμε τα 
				δεδομένα του πίσω με ασφάλεια. Επειδή το 
				αντικείμενο μπορει να καταχωρηθεί, να μπει και 
				να ξαναβγεί, πρέπει να είμαστε προσεκτικοί
			\item Αν ξεμείνουμε από πόρους (χώρο στο hash table, 
				buckets κτλ, πρέπει να συνεχίσουμε μονο όταν 
				μπορούμε
		\end{itemize}
	}
\end{frame}
