\begin{frame}[t,plain]
\titlepage
	\note[item]{Καλημέρα σας, ονομάζομαι Αλέξιος Πυργιώτης\\
		Θα σας παρουσιάσω τη διπλωματική μου με τίτλο:..}
	\note[item]{Συγκεκριμένα, στο πρώτο σκέλος της παρουσίασης θα μιλήσουμε για 
		το Archipelago, ένα distributed storage layer και το RADOS, ένα storage 
		backend του Αρχιπέλαγο.}
	\note[item]{Στο δεύτερο σκέλος, θα μιλήσουμε για την προσφορά της 
		διπλωματικής μας και συγκεκριμένα δυο επεκτάσεις του Αρχιπέλαγο:
		\begin{itemize}
			\item τον cached, δηλαδή τον caching μηχανισμό μας και κύριο 
				αντικείμενο της διπλωματικης
			\item το synapsed, ένα συμπληρωματικό εργαλείο που στόχος του είναι 
				να δώσει στον cached δικτυακές δυνατότητες
		\end{itemize}
	}
	\note[item]{Aν κάποια έννοια σας είναι άγνωστη ή για όποια απορία κατά τη 
		διάρκεια της παρουσίασης, μπορείτε ελεύθερα να με διακόψετε και να 
		ρωτήσετε.}
	
\end{frame}

\begin{frame}[t]{Contents}

	\tableofcontents

	\note[item]{O κορμός της παρουσίασης είναι ο εξής:
		\begin{itemize}
			\item Αρχικά, παρουσιάζουμε κάποια εισαγωγικά που 
				αφορούν το background της εργασίας μας.  
				Αναφέρουμε τι είναι το Synnefo και τι είναι η 
				υπηρεσια ~okeanos
			\item Έπειτα, μιλαμε για το archipelago, το RADOS και τελικά τον 
				στόχο της διπλωματικής.
			\item Στη συνέχεια μιλάμε για το τι είναι caching και 
				αναφέρουμε κάποιες σύγχρονες λύσεις για 
				caching.
			\item Τα επόμενα δυο κεφάλαια παρουσιάζουν τη σχεδίαση και 
				αξιολόγηση του cached.
			\item Αντίστοιχα παρουσιάζουμε το synapsed, τη σχεδίαση και την
				αξιολόγηση του
			\item Τέλος, συνοψίζουμε όσα ειπώθηκαν παραπάνω και μιλάμε για 
				μελλοντικές πιθανές μελοντικές εργασίες
		\end{itemize}
	}

\end{frame}

\section{Introduction}

\begin{frame}{Thesis background}
	\note{Ας ξεκινήσουμε με την παρούσα κατάσταση.\\
	Το software που τα ξεκίνησε όλα είναι το Synnefo}

	\includegraphics[height=0.075\textheight]{images/synnefo-logo.png}

	Open source, production-ready, cloud software.\\
	Designed since 2010 by GRNET.
	\note{\dspc..by GRNET -> Και φυσικά τα παιδιά που βλεπετε εδώ}
	\dspc
	\dspc
	\includegraphics[height=0.075\textheight]{images/okeanos-logo.png}

	\begin{itemize}
		\item IaaS service
		\item Targeted at the Greek Academic and Research Community
		\item Designed by GRNET
		\item In production since 2011
	\end{itemize}

	\note{
		\begin{itemize}
			\item IaaS είναι πρακτικά η παροχή εικονικής υποδομής σε χρήστες 
				(δηλαδή πάρε υπολογιστή (VM), δίκτυα, αρχεία κτλ)
			\item Φτιαγμένο επίσης από την ΕΔΕΤ
			\item Δωρεάν για τους Ακαδημαϊκους σκοπούς, ήδη χρησιμοποιείται ως 
				πλατφόρμα για εργαστήρια στο EMP και απ' αυτό το εξάμηνο σε 
				άλλες σχολές
		\end{itemize}
	}
\end{frame}
