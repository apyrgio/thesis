\chapter{Introduction}\label{ch:intro}

\section{Introduction/Motivation}

Bla-bla...

\begin{comment}
Also, excuse me in advance for the extensive use of the second person form in 
the following chapters, as well as breaking the fourth wall between the writer 
and the reader. Although I tried to, I find it difficult to communicate the more 
\textit{humane} aspect of this thesis such as code decisions, pitfalls and 
shortcomings.  After all, there is "you" who reads this thesis and "I" who 
writes it and no amount of third person form can cover that.
\end{comment}

\section{Thesis structure}

% TODO: Make `Chapter x` click-able and anchored to the respective chapter
\begin{description}
\item[Chapter~\ref{ch:ch2}:]
We define what "cloud" means and mention some of the most notable examples.
Then, we give a brief overview of the synnefo implementation, its key
characteristics and why it can have a place in the current cloud world.
\item[Chapter~\ref{ch:archipelago}:]
We present the architecture of Archipelago and provide the necessary theoretical
background (mmap, IPC) the reader needs to understand its basic concepts. Then,
we thoroughly explain how Archipelago handles I/O requests. Finally, we mention
what are the current storage mechanisms for Archipelago and evaluate their
performance.
\item[Chapter~\ref{ch:tiering}:]
We explain why tiering is important and what is the state of tiered storage at
the moment (bcache, flashcache, memcached, ramcloud, couchbase).  Then, we
provide the related theoretical background for cached (hash-tables, LRUs).
Finally, we defend why we chose to roll out our own implementation.
\item[Chapter~\ref{ch:cached-design}:]
We explain the design of cached, the building blocks that is consisted of
(xcache, xworkq, xwaitq). Then, we provide extensive benchmark results and
compare them to the ones of Chapter 3.
\item[Chapter~\ref{ch:ch6}:]
TODO
\item[Chapter~\ref{ch:ch7}:]
We draw some concluding remarks and propose some future work.
\end{description}
