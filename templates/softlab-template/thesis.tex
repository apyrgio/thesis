% DOCUMENT FORMAT ======================= -*- mode: LaTeX; coding: utf-8 -*- ===

\documentclass[diploma]{softlab-thesis}


% PACKAGE SETTINGS =============================================================

\usepackage[cm-default]{fontspec}
\usepackage{amsmath}
\usepackage{amsfonts}
\usepackage{multirow}
\usepackage{array}
\usepackage{mdwlist}
\usepackage{subfig}
\usepackage{floatrow}
%\usepackage{float}
\usepackage{verbatim}
\usepackage{color}
\usepackage{graphicx}
\usepackage{xunicode}
\usepackage{xltxtra}
\usepackage{url}
%\usepackage{dsfont}
%\usepackage{microtype}
\usepackage[htt]{hyphenat}
\usepackage{multicol}
\usepackage{wrapfig}

% FONT SETTINGS ===============================================================

%\setromanfont[Mapping=tex-text]{CMU Serif}
%%\setromanfont[Mapping=tex-text]{CMU Sans Serif} % temporary change until printing
%%\setsansfont[Mapping=tex-text]{CMU Sans Serif}
%%\setmonofont[Mapping=tex-text]{CMU Typewriter Text}
%\setmainfont[Mapping=tex-text]{CMU Serif}
%%\setmainfont[Mapping=tex-text]{CMU Sans Serif}  % temporary change until printing

%\setromanfont[Mapping=tex-text,ExternalLocation=fonts/]{cmunrm.otf}
%\setsansfont[Mapping=tex-text,ExternalLocation=fonts/]{cmunss.otf}
%\setmonofont[Mapping=tex-text,ExternalLocation=fonts/]{cmuntt.otf}
%\setmainfont[Mapping=tex-text, ExternalLocation=fonts/]{cmunss.otf}

% CUSTOM COMMANDS ==============================================================

% Layout macros
\newcommand\spa[1]{\; #1 \;}

% Font macros
\newcommand\resfont[1]{\ensuremath{\mathtt{#1}}}

% Mathematical macros
\newcommand\setmap[3]{#1\{#2 \mapsto #3\}}
\newcommand\getmap[3]{(#2 \mapsto #3) \in #1}
\newcommand\tuple[2]{\ensuremath{\langle#1, #2\rangle}}
\newcommand\mfrac[2]{\ensuremath{\dfrac{#1}{#2}}}
\newcommand\nequiv[2]{\ensuremath{#1 \not\equiv #2}}

% Core Ruby Operational Semantics letter bindings
\newcommand\mem{\mu}

% Core Ruby Operational Semantics low level macros
\newcommand\state[2]{(#1, #2)}
\newcommand\transition[1]{\ensuremath{\overset{#1, c*}{\rightarrow}}}
\newcommand\range[2]{#1, ..., #2}
\newcommand\midrange[5]{\range{#1}{#2}, #3, \range{#4}{#5}}

% Core Ruby Operational Semantics high level macros
\newcommand\operation[5]{\ensuremath{\state{#1}{#2} \transition{#3} \state{#4}{#5}}}
\newcommand\propagation[2]{\operation{#1}{\mem}{#2}{#1'}{\mem'}}

% Core Ruby specific Operational Semantics macros
\newcommand\semicolon[2]{#1; \; #2}
\newcommand\assign[2]{#1 = #2}
\newcommand\mcall[3]{#1.\texttt{#2}(#3)}
\newcommand\ifte[3]{\resfont{if} \; #1 \; \resfont{then} \; #2 \; \resfont{else} \; #3}
\newcommand\newclass[2]{#1.\resfont{new}(#2)}
\newcommand\methoddef[3]{\resfont{def} \; #1(#2) = #3}
\newcommand\classdef[2]{\resfont{class} \; #1 = #2}
\newcommand\with[3]{with \; \tuple{#1}{#2} \; do \; #3}

% Success Typing macros
\newcommand\ssub{\sqsubseteq_S}

% Core Ruby Success Typing letter bindings
\newcommand\classlist{\Delta}
\newcommand\envir{\Gamma}
\newcommand\fields{\Phi}
\newcommand\currclass{l}

% Core Ruby Success Typing inferencing macros
\newcommand\stinfer[5]{\classlist; \; #1; \; \fields \; \underset{\currclass}{\vdash} \; #2: #3 \; \& \; #4; \; #5}


%%%%%%%%%%%%%%%%%%%%%%%%%%% HASKELL STUFF %%%%%%%%%%%%%%%%%%%%%%%%%%%

%\newcommand\typerep[1]{\ensuremath{#1}}
\newcommand\typerep[1]{\lstinline[basicstyle=\normalsize\ttfamily,keywords={}]|#1|}
\newcommand\typefootrep[1]{\textbf{\lstinline[basicstyle=\footnotesize\ttfamily,keywords={}]|#1|}}
% \newcommand\ttyperep[1]{\typerep{#1}}
% \newcommand\mtyperep[1]{\mbox{\typerep{#1}}}

% Arrow types
\newcommand\typeto[2]{\typerep{#1} \typerep{->} \typerep{#2}}
\newcommand\typetotwo[3]{\ensuremath{\typerep{#1} \typerep{->}
                                     \typerep{#2} \typerep{->}
                                     \typerep{#3}}}

\newcommand\tyconapone[2]{\ensuremath{\mbox{\typerep{#1}} \:\: \mbox{#2}}}
\newcommand\tyconaponeC[2]{\ensuremath{\mbox{\typerep{#1}} \:\: \mbox{\typerep{#2}}}}
% \newcommand\tyconapone[2]{\typerep{#1} $\:$ \typerep{#2}}

\newcommand\tyconaptwo[3]{\ensuremath{\mbox{\typerep{#1}} \:\: \mbox{#2} \:\: \mbox{#3}}}


% FLOAT SETUP ==================================================================

% SPELLING =====================================================================

% CODE HIGHLIGHTING ============================================================

\usepackage{listings}

\definecolor{gray}{rgb}{0.5,0.5,0.5}
\definecolor{darkgreen}{rgb}{0.0,0.5,0.0}

\lstset{
	language=Haskell,
	basicstyle=\footnotesize\ttfamily,
	numbers=left,
	numberstyle=\tiny\color{gray},
	stepnumber=1,
	numbersep=5pt,
	backgroundcolor=\color{white},
	showspaces=false,
	showstringspaces=false,
	showtabs=false,
	frame=single,
	rulecolor=\color{black},
	tabsize=2,
	captionpos=b,
	breaklines=true,
	breakatwhitespace=false,
	title=\lstname,
        xleftmargin=17pt,
        xrightmargin=5pt,
        aboveskip=\baselineskip
}

\newcommand\haskellcode[2]{\lstinputlisting[language=Haskell,caption={#1},label=#1]{src/#2}}
\newcommand\rubycode[2]{\lstinputlisting[language=Ruby,caption={#1},label=#1]{src/#2}}
\newcommand\erlangcode[1]{\lstinputlisting[language=erlang,captionpos=]{src/#1}}
\newcommand\plaintext[1]{\lstinputlisting[language=,captionpos=,stepnumber=0]{src/#1}}

% CHANGE MATH FONT ============================================================

% HYPERREF MUST BE LAST =======================================================

\usepackage[xetex,colorlinks=true,linkcolor=blue,citecolor=darkgreen]{hyperref}

% DOCUMENT INFORMATION =========================================================

\title{Thesis subject} % ===============> FIXME
\author{Όνομα Φοιτητή}
\authoren{Student name}
\datedefense{9}{9}{9999}      % ===============> FIXME
\TRnumber{CSD-SW-TR-*-*} % ok ==================> FIXME
\supervisor{Υπέυθυνος Διπλωματικής}
\supervisorpos{Τίτλος Υπευθύνου}
\committeeone{Πρώτο μέλος επιτροπής}
\committeeonepos{Τίτλος μέλους}
\committeetwo{Δεύτερο μέλος επιτροπής}
\committeetwopos{Τίτλος μέλους}
\committeethree{Τρίτο μέλος επιτροπής}
\committeethreepos{Τίτλος μέλους}

\hypersetup{
	pdftitle={Thesis subject}, % ===> FIXME
	pdfauthor={Όνομα Φοιτητή},
	pdfsubject={},
	pdfkeywords={}
}


% MAIN DOCUMENT ================================================================

\begin{document}

\frontmatter
\maketitle

\def\templen{\parindent}
\setlength{\parindent}{0pt}
\setlength{\parskip}{1.5ex plus 0.5ex minus 0.2ex}
\begin{abstractgr}
	Η υπηρεσία αποθήκευσης δεδομένων μιας υποδομής υπολογιστικού νέφους 
	(cloud computing) έχει κατά κανόνα υψηλές απαιτήσεις σε επιδόσεις, 
	ειδικά όταν είναι επιφορτισμένη με την παροχή των εικονικών δίσκων για 
	τις Εικονικές Μηχανές (VMs) των χρηστών. Σε μεγάλες εγκαταστάσεις όμως, 
	παρατηρείται ότι η επίδοση των σκληρών δίσκων (hard disks) συνήθως δεν 
	επαρκεί για την εξασφάλιση γρήγορων ταχυτήτων για τις ανάγκες των 
	χρηστών. Αυτό οδηγεί τους προγραμματιστές στο να καταφεύγουν σε λύσεις 
	προσωρινής αποθήκευσης δεδομένων (caching) σε γρηγορότερα αποθηκευτικά 
	μέσα, για να βελτιώσουν τις επιδόσεις τους.
	
	Η παρούσα διπλωματική εργασία παρουσιάζει τον \textit{cached}, έναν 
	caching μηχανισμό ο οποίος δημιουργήθηκε για τη βελτίωση της επίδοσης 
	του Archipelago, ενός κατανεμημένου συστήματος αποθήκευσης σε 
	περιβάλλον υπολογιστικού νέφους (cloud computing environment).	
	Επιπροσθέτως, παρουσιάζεται το συμπληρωματικό δικτυακό εργαλείο 
	\textit{synapsed}, το οποίο ανοίγει το δρόμο για την αναβάθμιση του 
	\textit{cached} σε έναν πλήρως κατανεμημένο caching μηχανισμό.
	
	Τα αποτελέσματα από τις πρώτες μετρήσεις είναι αρκετά ενθαρρυντικά και 
	δείχνουν ότι η χρήση του \textit{cached} μπορεί να αυξήσει την επίδοση του 
	Archipelago μέχρι και 400\%.
	\begin{keywordsgr}
		cached, synapsed, synnefo, okeanos, rados, archipelago, cloud, εικονική 
		μηχανή, προσωρινή αποθηκευτική μνήμη, Ο(1) πολυπλοκότητα, φυσική μνήμη, 
		κατανεμημένο σύστημα
		\end{keywordsgr}
\end{abstractgr}

\begin{abstracten}
	The performance of the storage service of a cloud infrastructure is very 
	critical, especially when it provides virtual volumes for the Virtual
	Machines of users. Underlying hard disks however, are usually inadequate in 
	providing fast performance for large deployments and the storage engineers 
	commonly have to resort in using various buffering and caching techniques.
	
	This diploma thesis presents \textit{cached}, an in-memory caching 
	mechanism that aims to improve the performance of Archipelago, a 
	software-defined, distributed storage layer for cloud computing 
	environments.  Additionally, this diploma thesis introduces 
	\textit{synapsed}, a complementary network tool that paves the way for 
	the creation of a fully distributed cache.
	
	Early performance evaluations look very promising and show that 
	\textit{cached} can improve the current Archipelago performance up to 
	400\%.
	\begin{keywordsen}
		cached, synapsed, synnefo, okeanos, rados, archipelago, cloud, 
		virtual machine, volume service, cache, hash table,  O(1) 
		complexity, ram, distributed system, replication
	\end{keywordsen}
\end{abstracten}

\begin{acknowledgementsgr}
	\todo
\end{acknowledgementsgr}


\selectlanguage{english}   % Everything from here on is in english
\setlanguage{english}
\nogreekalph

\setlength{\parindent}{\templen}
\setlength{\parskip}{0pt}
\tableofcontents
\listoffigures
%\listoftables
\renewcommand{\lstlistlistingname}{List of Listings}
%\lstlistoflistings % changed the title above

\mainmatter
% moved these two commands here so that they don't influence the toc
\setlength{\parindent}{0pt}
\setlength{\parskip}{1.5ex plus 0.5ex minus 0.2ex}

\renewcommand\floatpagefraction{.7}

\chapter{Chapter 1}\label{ch:ch1}

\section{Section 1}

First section.

\section{Section 2}

Second section.
It also includes a minor citation\cite{misc_citation}.

\section{Outline}

You can choose between two types of outlines. This:

Chapter~\ref{ch:ch2} does this and this. Chapter~\ref{ch:ch3} does that and
that. Chapter~\ref{ch:ch4} does these and these. And finally,
chapter~\ref{ch:ch5} does those and those.

or this:

\begin{description}
\item[Chapter]~\ref{ch:ch2}
	We do this and this.
\item[Chapter]~\ref{ch:ch3}
	We do that and that.
\item[Chapter]~\ref{ch:ch4}
	We do these and these.
\item[Chapter]~\ref{ch:ch5}
	We do those and those.
\end{description}

\chapter{Chapter 2}\label{ch:ch2}

\section{Section 1}

This section has an important citation\cite{inproceeding}

\subsection{Subsection 1}

This subsection has code in Haskell:
\haskellcode{Sample code}{samplecode.hs}

It also has a list:
\begin{description}
\item[Item 1]
	First item
\item[Item 2]
	Second item and a footnote\footnote[1]
	{Footnote description.}.
\item[Item 3]
	Third item and text in \textit{italics}.
\end{description}

And an enumerated list:

\begin{enumerate}
\item	First item.
\item	Second item and text in \textbf{bold}
\end{enumerate}

\section{Section 2}

\subsection{Subsection 1}
This subsection has a link to the block of code~\ref{Sample code} in Section 1.

\subsection{Subsection 2}

% FIXME: Must add content here
This subsection has a FIXME comment, visible only to the author.

\chapter{Chapter 3}\label{ch:ch3}

\section{Section 1}

This how we add a url: \url{http://www.example.org}

\section{Section 2}

And this is how we point to Figure~\ref{fig:image1}.

\begin{figure}[hb]
	\centering
	\includegraphics[]{diagrams/9.jpg}
	\caption{This is an image}
	\label{fig:image1}
\end{figure}

\subsection{Sub-section 3}

Another way to create a list:
\begin{itemize}
	\item Item 1
	\item Item 2
	\item Item 3
	\item Item 4
\end{itemize}


\chapter{Chapter 4}\label{ch:ch4}

\section{Section 1}

We can also use a special fonts to differentiate between text and $math()$.


\chapter{Chapter 5}\label{ch:ch5}

\section{Section 1}

This is an equation:
\begin{equation}
	true \in 9
\label{rule:9}
\end{equation}

which can once again be referenced like so \ref{rule:9}.


\backmatter
\cleardoublepage % start at the next odd page
\phantomsection % correct hyperlinking
\bibliography{references}
\bibliographystyle{softlab-thesis}
% \include{glossary}
% \chapter{Appendix}
% \printindex

\end{document}
