\chapter{Chapter 3}\label{ch:archipelago}

\section{Necessary theoretical background}

\subsection{Multi-threaded programming}

% http://accu.org/index.php/journals/1634

Multi-threading programming is good and is bad and here are some challenges:

\begin{enumerate}
	\item Concurrency control
	\item Challenge 2
	\item Challenge 3
\end{enumerate}

\subsubsection{Concurrency control}

\paragraph{Locking}

% Taken from wikipedia

Three concepts for locking:

\begin{enumerate}
	\item Lock overhead
	\item Lock contention
	\item Deadlocking
\end{enumerate}

\subsection{IPC}

Below we can see some IPC methods:

\begin{enumerate}
	\item mmap()
	\item Semaphores
	\item Sockets
\end{enumerate}

\section{Archipelago}

Archipelago consists of the following:

1. XSEG
2.
3.

\section{XSEG}

XSEG is the segment on which the IPC...

There are some XSEG stuff such as:

1. Drivers
2. Libraries
3. Xtypes
4. Peers

\subsection{Drivers}

\subsection{Libraries}

\subsection{Xtypes}

The rationale behind xtypes is:

\begin{itemize}
	\item Abstraction(?) layers: Creating inner abstractions layers for 
		software is not a new concept but it's very easy to miss, 
		especially when you start small and end up big.
		
		In a nutshell, when writing code for a new software (in our case 
		a peer for Archipelago but this can apply to most software that 
		surpass the 1000 LOC\footnote[1]
		{Lines Of Code}
		mark) it is wrong practice to create from scratch a monolithic 
		implementation with indistinguishable parts. There is a main 
		reason for this:
		
		Monolithic implementations usually derive from lack of code 
		architecture and planning. Although it is feasible for a 
		programmer to create fully-functional code that meets the 
		necessary requirements, albeit with a lot more effort and 
		concentration, this approach will backfire when the programmer 
		needs to add new features. Since there is no explicit code 
		architecture and the fragile inner correlations are between 
		lines of code and not separate entities, stored precariously in 
		the developer's mind, the result will eventually be constant 
		code refactorization.
		
		One might think that new features happen once in a while in the 
		development cycle but that would be wrong.  This happens more 
		often than you might think and is actually the common case in 
		iteration and test-driven development.

		The right practice instead is to...
	\item Re-usability:...
	\item User-space / Kernel-space agnosticity: (I doubt that such a word 
		even exists...)
\end{itemize}	

\subsection{Peers}
