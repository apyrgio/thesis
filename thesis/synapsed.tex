\chapter{The Synapsed peer}

On the previous sections, we have evaluated the design of cached and we have 
seen that is heavily bounded by CPU. The implications of this, however,
are bigger, if we consider that all the Archipelago is running in the host 
machine, whose CPU's are already oversubscribed. This means that cached must 
compete for CPU time against the running VMs, essentialy defeating the QoS 
purpose it serves

On the other hand, on the RADOS nodes, the CPUs are not used to the full 
potential. Thus, it would be interesting to check how well cached (and 
Archipelago in general) would behave, if the VM's data where send initially 
over network and then handled by the vlmc. So what we mean is to turn the 
current situation from how it is in Figure ? to the following:

\fixme add figure for network

To this end, we have created a network peer called synapsed (from 
\textbf{synapse d}aemon) as a proof-of-concept, that should be able to receive 
any type of requests and send them through the network to another Archipelago 
segment.

More specifically, on Section...

\section{Design of synapsed}

Given that currently Archipelago is not network-aware, peers from one segment 
cannot know the ports of peers of another segment. Moreover, they cannot send a 
request to the synapsed peer and simultaneously target a peer in another 
segment. So, we are faced with the problem of what will be the limitations of 
synapsed.

We propose the following solution to this problem: Each of those two segments 
must have at least one synapsed peer. So, when one synapsed accepts an XSEG 
request, it will translate it to a network request, send it to the synapsed 
peer of the other segment which will finally translate it back to an XSEG 
request.  Moreover, each synapsed peer will be attached to a peer of its 
segment. Thus, the re-translated XSEG request will only be send to that target. 

This means that synapsed does not actually connect two remote segments
but bridge two remote peers over network.

Moreover, an important issue is how the requests will be send over network. Due 
to the above limitation, each synapsed will handle no more than one connection




Thus, the synapsed needs the following arguments:

\begin{enumerate}
	\item Its port where it will accept connections
	\item Its target port where

