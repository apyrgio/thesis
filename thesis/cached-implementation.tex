\chapter{Implementation of cached}\label{ch:cached-implementation}

In the previous chapter, we presented a design overview for cached and its 
components. In this chapter we will blabla how the above design has been
implemented and explain in depth the structures and functions that have been 
created for this purpose.

More specifically, sections ? - ? provide implementation information for the 
components of cached, as described in Chapter ?. Next, section ? presents the 
actual initialization and blabla operations using excerpts from the code.

\section{Implementation of xcache}

In this section, we describe how we implemented the design concept of section 
\ref{sec:xcache-design}. The main \xcache structure is the following:

\ccode{Main \xcache struct}{xcache-struct.h}

Each of the above \xcache struct fields serves a design purpose.
Let's see which fields help in what:

\subsection{Entry Preallocation}

The relevant fields for this purpose can be seen in the following code listing:

\ccode{\xcache struct fields for preallocated entries}{xcache-prealloc.h}

and the definition of the \texttt{xcache entry} struct which shows up in 
\texttt{xcache} struct can be seen below:

\ccode{\texttt{xcache entry} struct}{xcache-entry.h}

Let's start by listing what \texttt{xcache entry} consists of. First of all, it 
must have a name. Since we preallocate the entries and cannot know in runtime 
their length, we must allocate as much space as possible. The \texttt{char 
	name[XSEG\_MAX\_TARGETLEN + 1]} field, which is 256 characters long, is 
long enough to hold the target's name. Also, as we have mentioned in Section 
\ref{sec:entry-prealloc-design}, xcache must be agnostic of the cache contents.  
To this end, we use the generic \texttt{void *priv} field as a pointer to the 
actual entry content. The rest of the fields will be explained in the following 
chapters.

Let's continue now with the fields of Listing \ref{lst:xcache-prealloc.h}. Since 
we preallocate the entries using \texttt{malloc}, they take up a contiguous 
space in memory.  The start of this space is the where the \texttt{*nodes} field 
points to. The \texttt{free\_nodes} field works similarly to the free\_entries 
field in Section \ref{sec:get-req-archip} i.e. it is a stack where indexes to 
unused nodes are pushed.

\subsection{Entry Indexing}

The relevant fields for this purpose can be seen in the following code listing:

\ccode{xcache struct fields for entry indexing}{xcache-index.h}

As we have mentioned in Section \ref{sec:xcache-index-design}, we utilize two 
hash tables, one for the cached entries and one for the former cached entries 
(or evicted entries or removed entries). These hash tables can be accessed from 
the \texttt{xcache struct} and are \texttt{*entries} and \texttt{*rm\_entries} 
respectively.

\subsection{Entry eviction}

The relevant fields for this purpose can be seen in the following code listing:

\ccode{\xcache struct fields for eviction}{xcache-evict.h}

As we have mentioned in Section \ref{sec:xcache-evict-design}, we resort to 
eviction when the cache is full and new entries can't be inserted. This entry is 
the Least Recently Used entry. The doubly-linked list we maintain for this end 
can be seen below:

\ccode{Doubly-linked LRU list}{xcache-dlist.h}

Lets explain these fields a bit:

\begin{description}
	\item[lru:] Obviously, the least recently used entry. It can be 
		considered as the one end of the doubly linked list.
	\item[mru:] The entry that has just been used. It can be considered as 
		the other end of the doubly-linked list
	\item[younger:] This entry specific field points to an entry used right 
		after our entry was used.
	\item[older:] Same as "younger", it points to the entry that has been 
		used right before our entry was used.
\end{description}

\subsection{Concurrency control}\label{sec:conc-imp}

\paragraph{Locking}\label{par:lock}

\paragraph{Reference counting}\label{par:refcount-imp}

The refcount model in xcache should be familiar to most people:

% Turn this to figure
\begin{itemize}
	\item When an entry is inserted in cache, the cache holds a reference 
		for it (ref = 1).
	\item Whenever a new lookup for this cache entry succeeds, the reference 
		is increased by 1 (ref++)
	\item When the request that has issued the lookup has finished with an 
		entry, the reference is decreased by 1. (ref--)
	\item When a cache entry is evicted by cache, the its ref is decreased 
		by 1. (ref--)
\end{itemize}

Some common refcount cases are:

\begin{itemize}
	\item active entry with pending jobs (ref > 1)
	\item active entry with no pending jobs (ref = 1)
	\item evicted entry with pending jobs (ref > 0)
	\item evicted entry with no pending jobs (ref = 0)
\end{itemize}

\begin{table}[tbp]
	\centering
	\begin{tabular}{ | l | l | }
		\hline
		Case & Refcount \\ \hline \hline
		active entry with pending jobs & ref > 1 \\ \hline
		active entry with no pending jobs & ref = 1 \\ \hline
		evicted entry with pending jobs & ref > 0 \\ \hline
		evicted entry with no pending jobs & ref = 0 \\ \hline
	\end{tabular}
	\caption{Reference counting of xcache}
	\label{tab:refcount}
\end{table}

and, as always, the entry is freed only when its ref = 0.

