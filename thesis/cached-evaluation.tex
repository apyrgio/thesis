\chapter{Performance evaluation of cached}\label{ch:cached-evaluation}

\begin{flushright}
	\textit{"There are three kinds of lies:\\
		lies, damned lies, \\
		and \sout{statistics} benchmarks."}	\\

	Mark Twain (modernized)
\end{flushright}

It may seem as an ironic statement, considering that we are about to provide 
benchmark results for cached, but it is actually a valid one.
\begin{comment}
What Mr.  Twain tries to say here
\footnote{
	and that's a phrase usually not heard in programming contexts...
}
is that the presentation of partials facts for something can be used to 
fabricate a plausible truth for it.
%This is shit, fix it
In science's case, it so often happens that promising results for an experiment 
can seem more important to the researcher's eye than negative ones due to 
positive reinforcement.
\end{comment}
In our case, we will try not to smear the next pages with diagrams but first 
explain the benchmarking methodology behind them, provide a concrete depiction 
of cached's performance under various workloads, and finally explain the 
results in depth.

The skeleton of this chapter is the following:

Section \ref{sec:perf-meth} explains the methodology behind our measurements.  
Section \ref{sec:test-bed} provides details about the hardware on which we have 
conducted our benchmarks.  Section \ref{sec:vs-plot} presents comparative 
benchmarks for cached and sosd and quantify the performance gain of our 
implementation. Section \ref{sec:internals-plot} sheds some light on the 
internals of cached with the use of several special-purpose synthetic 
benchmarks.  Finally, Section \ref{sec:vm-plot} evaluates the actual 
performance of Archipelago and cached for a real VM and compares the results 
with the results of the previous sections.

\section{Benchmark methodology}\label{sec:perf-meth}

The benchmarks that have been executed and whose results are presented in this 
chapter, will be split in three categories, all of which have their own 
distinct goals:

The first category is the comparison between using cached \textbf{on top} of 
the sosd peer (sosd has been discussed in Section \ref{sec:rados-archip}) and 
using solely the sosd peer.

\begin{comment}
The category's goal is to "defend" one of the core thesis arguments, that 
tiering is a key element that will improve the performance of Archipelago.  
\end{comment}

In order to effectively compare the performance of cached and sosd, we must 
consider the following: 

\begin{enumerate}
	\item The comparison of the two peers should try to focus on what is 
		the best performance that these peers can achieve for a series 
		of tough workloads.
	\item The circumstances under which both peers will be tested need not 
		be thorough but challenging. For example, it may be interesting 
		to test both peers against sequential requests, but
		\begin{inparaenum}[i)]
		\item such patterns are rarely a nuisance for production environments,
		\item they do not stress the peers enough to provide something 
			conclusive and
		\item they are out of the scope of this section as there can be many 
			tests of this kind and adding them all here will impede the 
			document's readability.
		\end{inparaenum}
	\item Both peers must be tested under the same, reasonable workload, i.e a 
		workload that can be encountered in production environments.
	\item If a peer doesn't show a consistent behavior for a workload, it must 
		be depicted in the results
\end{enumerate}

Having the above in mind, the next step is to choose a suitable workload.  This 
choice though is fairly straight-forward; in production environments, the most 
troublesome workload is the burst of small random reads/writes and is usually 
the most common one that is benchmarked.  

One may ponder however, how many requests can be considered as a "burst" or 
which block size is considered as "small". Of course, there is not only one 
answer to this question so, we will work with ranges. For our workload, we will 
use block sizes ranging from 4KB to 64KB and parallel requests ranging from 4 
to 16.

The second category deals solely with the inner-workings of cached and its 
behavior on different operation modes or states. Its aim is not to capture the 
performance against a tough workload, but to explain \textbf{why} this 
performance is observed and how each of the options affect it. In this 
category, we measure how threads impact the performance of cached or what 
impact (if any) does our index mechanism have.

The third category positions cached and sosd in a real-world scenario; an 
actual VM that is configured to have Archipelago as its backing storage and on 
which we conduct the same benchmarks as in the previous categories. Our aim in 
this category is to:

\begin{enumerate}
	\item Measure under real circumstances the performance of the cached 
		configurations that have been discussed in the previous 
		scenarios.
	\item Compare their results with the respective results of the previous 
		categories.
\end{enumerate}

This way, we will be able to check if the reported performance gain of the 
previous categories can be applied in a real-world scenario.

Finally, in the following sections, for brevity reasons, we will talk about 
comparing cached and sosd. What the reader must keep in mind however is that 
cached is essentially the cache layer above sosd, which means that we actually 
test sosd vs cached over sosd.

\section{Specifications of test-bed}\label{sec:test-bed}

The specifications of the server on which we conducted our benchmarks is the 
following.

\begin{table}[H]
	\centering
	\begin{tabular}{ | l | l | }
		\hline
		Component & Description \\ \hline \hline
		CPU &  2 x Intel(R) Xeon(R) CPU E5645 @ 2.40GHz \cite{e5645} \\
		 & Each CPU has six cores with Hyper-Threading enabled, which equals to 
		 24 threads. \\ \hline
		RAΜ & 2 banks x 6 DIMMs PC3-10600 \\
		& Peak transfer rate: 10660 MB/s \\ \hline
	\end{tabular}
	\caption{Test-bed hardware specs}
	\label{tab:hardware-specs}
\end{table}

\begin{table}[H]
	\centering
	\begin{tabular}{ | l | l | }
		\hline
		Software & Version \\ \hline \hline
		OS &  Debian Squeeze \\ \hline
		Linux kernel & 3.2.0-0 (backported) \\ \hline
		GCC & Debian 4.4.5-8 \\ \hline
	\end{tabular}
	\caption{Test-bed software specs}
	\label{tab:software-specs}
\end{table}

\section{Performance comparison between cached and sosd}
\label{sec:vs-plot}

As mentioned above, for our first test, we will evaluate the read and write 
performance of cached and sosd for a random workload with parallel requests of 
small size. In order to measure accurately their performance, we will use two 
different metrics:

\begin{description}
	\item[Bandwidth:] \hfill \\
		Bandwidth measures the maximum throughput that the application 
		can sustain. This metric is usually used to indicate how much 
		I/O (from various inputs) can an application handle within a 
		second.
	\item[Latency:] \hfill \\
		Latency is the converse of bandwidth. It is a measurement from 
		the viewpoint of the issuer of requests and indicates the 
		responsiveness of the tested application. It is commonly 
		calculated as the average reply time for a series of requests.
\end{description}

On the following sections, we present the benchmarks that we conducted for the 
first category. The first benchmark, which is shown in Section 
\ref{sec:peak-plot}, shows the behavior of cached during a peak of I/O 
requests.  The second benchmark, which is shown in Section 
\ref{sec:sustained-plot}, illustrates the behavior of cached under continuous 
load.  On these sections, cached is always multi-threaded and uses 4 threads.

\subsection{Workload smaller than cache size - Peak behavior}
\label{sec:peak-plot}

We begin with the bandwidth performance of our peers. The write performance can 
be seen in Figure \ref{fig:bw-write-comp-lie.pdf}, while the read performance 
can be seen in Figure \ref{fig:bw-read-comp-lie.pdf}.

Before we proceed with the interpretation of the diagram results, we will 
briefly comment on the diagram structure. Due to the fact that the performance 
of the two peers differs in at least two orders of magnitude, the results would 
look too flat in a conventional diagram that would scale from 0 to 11000. To 
amend this, we have broken the y-axis of our diagrams in two parts with 
different scales and starting values, in order to make the comparison easier to 
the eye.

\diagram{Comparison of bandwidth performance for write 
	peaks}{bw-write-comp-lie.pdf}
\diagram{Comparison of bandwidth performance for read 
	peaks}{bw-read-comp-lie.pdf}

The initial results look very promising. For write requests, the speedup for 
very small block sizes (4KB - 16KB) is approximately \textbf{100x} whereas for 
larger ones (32KB - 64KB) it ranges from \textbf{50x to 200x}. For read 
requests, the speedup for very small block sizes is approximately \textbf{50x}, 
whereas for larger block sizes, it ranges between \textbf{20x - 75x}.

These results not only illustrate the performance gap between RAM and
HDDs but also show that our implementation manages to keep its bandwidth 
consistently over 1 GB/s in stress scenarios. However, we must keep in mind 
that we see here is only a part of the full picture, since we do not know yet 
how cached behaves past its cache size.

Moreover, upon closer inspection the following questions arise:

\textbf{1) Where is the speedup difference between reads and writes attributed 
	to?}

We have mentioned above that the speedup for writes is between 50x and 200x 
whereas for reads is between 20x and 75x. Although these speedups are very big, 
they also show some of the limitations of our implementation.

We attribute the speedup differences to three factors:

\begin{enumerate}
	\item The performance of cached is almost the same for writes and reads.  
		\footnote{To be fair, reading seems a bit faster than writing, but that 
			is probably attributed to CPU caches and the fact that reading from 
			RAM is a non-destructive operation in contrast to writing.}
		This is expected behavior as the read and write paths for cached have 
		many common parts, as seen in Section \ref{sec:cached-flow-design}.
	\item Cached doesn't scale much past the 16KB block size. This is an 
		interesting observation with an unexpected answer. It may seem 
		implausible at first, but what happens is that we are actually hitting 
		the bandwidth limit of the RAM modules. If we consult the Table 
		\ref{tab:hardware-specs}, we can see that the bandwidth limit of our 
		RAM is about \mytilde10.7GB/s
		\footnote{A more careful look shows that we surpass this limit, 
			which is expected given the fact that we have a multi 
			channel RAM setup.}.
		This limit is approached asymptotically as the block size increases and 
		the index and concurrency overhead decreases
	\item On reads, sosd is benefited from the existence of caches in various 
		levels e.g. on OSD level, on RAID controller level.
\end{enumerate}

To sum up, the cached's performance remains relatively the same in both reads 
and writes, it is merely the sosd that is getting faster in reads due to 
caches. 

\textbf{2) Why is cached's performance increased proportionally to the block 
	size?}

We expected that sosd's performance would increase proportionally to the block 
size, due to the rotational nature of hard disk drives, but why does this 
affect cached too? It is certainly not attributed to any \texttt{memcpy()} 
performance tricks, since we always write in bucket granularity, which means 
that a 16KB write is translated to 4 x 4KB writes.

From this observation, we extrapolate that the CPU operations such as indexing,
job enqueuing, accepting and responding requests, that occur proportionally 
more times for smaller block sizes, dominate the cached's performance.

We can see this more clearly if we consult Table \ref{tab:2threads}.

\begin{table}[h!]
	\centering
	\begin{tabular}{ | l | l | l | l | l |}
		\hline
		Bandwidth (MB/s) & IOPS	  & Latency (usec) & Iodepth& Block size (KB) \\ \hline \hline
  2652.327   &	678995.620& 5.002&	4&		4 \\ \hline
  4427.89l   &	566769.888& 6.176&	4&		8 \\ \hline
  5900.719   &	377646.042& 9.672&	4&		16 \\ \hline
  6895.167   &	220645.338& 17.179&	4&		32 \\ \hline
  7090.661   &	113450.584& 34.237&	4&		64 \\ \hline

	\end{tabular}
	\caption{Write performance results for 2-threaded cached}
	\label{tab:2threads}
\end{table}

On this table, we present the results for a benchmark that does writes, has 
four parallel write requests and its block size varies between 4KB and 64KB.  
Note that this benchmark is conducted on a cached that has \textbf{two} threads 
instead of four.

Comparing the results of this table with the results of Figure 
\ref{fig:bw-write-comp-lie.pdf}, and more specifically the "cached-iodepth4" 
trend line, we can observe two things:

\begin{enumerate}
	\item The performance for small block sizes (4KB - 8KB) is better than the 
		performance of cached with 4 threads.
	\item The gap between the results for 4KB and 8KB is not 2x, as in Figure 
		\ref{fig:bw-write-comp-lie.pdf}, but 1.5x.
	\item The overall performance gap between the results for 4KB and 64KB is 
		less than the gap in Figure \ref{fig:bw-write-comp-lie.pdf}.
\end{enumerate}

To sum up, these results show that the lock contention in cached, due to the 
number of threads, can significantly dominate its performance.

\textbf{3) Why cached's perforance does not improve proportionally to the 
	parallel requests?}

The reason why we see only a minor increase in the performance of cached, even 
though it is multi-threaded, is because our locking scheme is not fine-grained 
enough. We have a single lock for our request queue, a single lock for most of 
the hash table accesses and this inevitably causes a lot of threads to spin.  
This slight improvement we see is mainly due to the fact that requests are 
effectively being pipelined while waiting each other to release its lock.  

\diagram{Comparison of latency performance for write 
	peaks}{lat-write-comp-lie.pdf}
\diagram{Comparison of latency performance for read 
	peaks}{lat-read-comp-lie.pdf}
We now proceed to the latency results. The write performance can be seen in 
Figure \ref{fig:lat-write-comp-lie.pdf} while the read performance can be seen 
in Figure \ref{fig:lat-read-comp-lie.pdf}.

The latency results corroborate the observations that we made for the bandwidth 
results. As we can see, the latency of cached is proportional to the number of 
parallel requests, which reinforces the assumption that our locking scheme is 
not granular enough, whereas sosd has only a small variance.

\subsection{Workload larger than cache size - Sustained behavior}
\label{sec:sustained-plot}

We now proceed to the second part of the comparison between cached and sosd.  
On this part, we will once again evaluate their performance against a random 
workload with many parallel requests. Unlike the first part though, where the 
cache size was larger than the workload, on this part the cache size will
only be half of it. Having a smaller cache than the workload is after all the 
most common scenario for production environments and the results should 
complete the picture of cached's behavior in a performance-intensive 
environment.

For this test, there are two main parameters we must take into account: the 
cache size and the maximum objects. These parameters have been decoupled in our 
implementation and we expect different results for each combination. More 
specifically, we have tested cached with half the workload size and:

\begin{enumerate}
	\item half the objects of the workload (referenced as 
		\textbf{cached-limited} from now on).
	\item more than the objects of the workload (referenced as 
		\textbf{cached-unlimited} from now on).
\end{enumerate}

Also, we have tested cached in write-through mode, to see its behavior in 
sustained writes. For this mode, cached has been configured to store more than 
the objects of the workload (i.e. unlimited) but still has space for the half 
of it.

Furthermore, for this benchmark, we tested only the write performance of cached 
and sosd, since the evictions and subsequent flushes in reads would skew
considerably the results. An exception to this is the write-through test, whose 
flushes do not produce extra writes to the storage backend.
Also, we chose for our workload to use 16 parallel requests and 4KB block size, 
since this seemed the most troublesome workload in the previous benchmark.

Before we proceed to the results, we must explain first how the diagrams below 
are structured. Since cached's performance is fairly unstable in this scenario, 
we have chosen to illustrate it as the benchmark progresses.  Thus, the x-axis 
shows what percent of the benchmark has been completed and for every 5\%, we 
show the performance of cached for that part.

The bandwidth results can be seen in Figure \ref{fig:bw-write-comp-truth.pdf}, 
while the latency results can be seen in Figure 
\ref{fig:lat-write-comp-truth.pdf}.  Due to the vast bandwidth drop of cached, 
the y-axis uses a logarithmic scale to show a more clear picture of what 
happens when the performance reaches the 10 MB/s mark.

\diagram{Comparison of bandwidth performance for sustained writes} 
{bw-write-comp-truth.pdf}

\diagram{Comparison of latency performance for sustained writes}
{lat-write-comp-truth.pdf}

These diagrams have several interesting points which we will address below.

\textbf{1) Why does cached-unlimited's performance drop in the middle of the 
	benchmark?}

Since cached-unlimited can index more objects than the workload, there is no 
need to evict entries from the hash table to insert new ones. Cached-limited on 
the other hand, has at any point a 50\% chance to receive a request for a 
target that does not index. Thus, when cached-unlimited starts to receive the 
first 50\% of requests, it only needs to store them and, as we have seen in the 
first part of this section, it does so very fast.

However, at around 50\% of the benchmark its space is depleted and thus, it has 
to manually evict entries. At this point, we see that its performance 
degenerates to the performance of cached-limited.

\textbf{2) Why is cached-limited's performance significantly less than the 
	sosd's performance?}

As we have explained above, cached-limited constantly evicts entries due to its 
object limitation. It may seem that the only cost of eviction is that the 
performance degenerates to the performance of sosd, but that is not entirely 
true.

When an entry is evicted, all of its contents must be flushed before the new 
entry can overwrite them. Moreover, the cost of concurrency control for hash 
table migrations and the cost of creating new requests and copying the data in 
them is not negligible. Finally, due to constant evictions, there are only a 
few buckets cached for any entry, which in turn does not leave any margin for 
coalescence.

\textbf{3) Why does cached-unlimited's performance increase consistently after 
	the 50\% mark ?}

After the 50\% mark, cached-unlimited will have to resort to evictions in order 
to store new data. Unlike cached-limited though, cached-unlimited has already 
cached half of the workload's data, which not only is favorable for 
coalescence, but also allows flushes to occur in a more sequential fashion.  
Moreover, cached evicts whole objects to get their buckets, which means that 
evictions will be less frequent that cached-limited, which constantly evicts 
entries to index other entries.

\textbf{4) Why does cached in write-through mode perform the same as sosd?}

The performance of cached in write-through mode is the same as sosd's since 
write requests are essentially forwarded to sosd and are cached when sosd 
responds. The caching has no significant overhead, compared to the waiting time 
for sosd, since there are no dirty entries and thus no costly evictions.

What is more interesting however is the comparison of read performance between 
cached and sosd, which can be seen in Table \ref{tab:writethrough}.

\begin{table}[H]
	\centering
	\begin{tabular}{ | l | l | l | l | l | l |}
		\hline
		Peer& Bandwidth (MB/s) & IOPS	& Latency (usec) & Iodepth & Block size 
		(KB) \\
		\hline \hline
		sosd& 34.124&		8735.699&		1828.758&	16&		4 \\ \hline

		cached& 44.897&		11493.511&		1389.511&	16&		4 \\ \hline

	\end{tabular}
	\caption{Read performance comparison of sosd and cached in 
		write-through mode}
	\label{tab:writethrough}
\end{table}

The read performance of cached has a noticeable increase of 25\%. This is 
expected, since cached serves read requests either from its cached data or from 
sosd. This means that it has the baseline read performance of sosd plus the 
extra performance gain from cache hits.

\section{Performance evaluation of cached internals}\label{sec:internals-plot}

To some extent, we have already discussed about the behavior of cached when we 
tweak various of its parameters, such as the over-subscription of its objects, 
the cache size or its write policy. In this category, we will further solidify 
our previous assumptions about the inner workings of cached by measuring:

\begin{enumerate}
	\item how cached scales with the number of threads
	\item the overhead of the index mechanism
\end{enumerate}

Note that the tests will be run with the following parameters:

\begin{description}
	\item[Mode] Write-back
	\item[Block size] 4KB
	\item[Cache size] Always larger than benchmark size
	\item[Operation] Writes
\end{description}

The above options have been chosen to isolate cached of any other factors that 
may alter its performance, such as evictions and delays due to sosd.

\subsection{Cached performance per number of threads}

Measuring the performance impact of multi-threading is nonsensical, if the 
application is not tested against many parallel requests. Hence, we will test 
each n-threaded cached configuration against a range of parallel requests and 
measure its performance.

The bandwidth results can be seen in Figure \ref{fig:bw-write-threads.pdf} 
whereas the latency results can be seen in Figure 
\ref{fig:lat-write-threads.pdf}.

\diagram{Bandwidth performance of cached per number of 
	threads}{bw-write-threads.pdf}
\diagram{Latency performance of cached per number of 
	threads}{lat-write-threads.pdf}

From these results, we derive the following conclusions:

\begin{enumerate}
	\item Cached is generally benefited from multi-threading, when the amount 
		of parallel requests is more than two. More specifically, we 
		achieve a major performance improvement of up to 40\% when 
		using two threads in contrast to one, as well as a lower, but 
		significant, performance improvement for up to four threads, as 
		the number of parallel requests increases.
	\item Cached does not scale past four parallel requests, regardless of 
		the number of threads.
	\item Adding more than two threads degrades significantly the performance 
		of cached, when the number of parallel requests is small.
\end{enumerate}

Finally, these results, along with the results of the first part, clearly show 
that we cannot fully utilize the multi-threading capabilities of cached unless 
we employ a more fine-grained locking scheme.

\subsection{Cold cache vs Warm cache}

This scenario will attempt to evaluate the overhead of cache misses for 
\textbf{write} operations. Theoretically, this should account to the overhead 
of adding new entries to cached and consecutively, an indication of the 
complexity of our index mechanism.

Our experimental procedure is the following:

\begin{enumerate}
	\item We inserted 128K
		\footnote{where \textbf{K} is \textbf{x1024} objects}, 256K, 512K and 
		1M
		\footnote{where \textbf{M} is \textbf{x1024\textsuperscript{2}} 
			objects}
		objects in a cached peer configured to hold up to 2M objects 
		and measured their average latency times (cold cache).
	\item Subsequently, we attempted to insert these objects in cached 
		again and measured their average latency times (warm cache).
\end{enumerate}

The difference between the first and the second iteration is that in the second 
iteration, the objects are hot in the cache and thus cached will not need to 
insert them. This means that the first iteration has the extra actions of 
allocating the entries and inserting them in the cache, whereas the second 
iteration will only lookup the objects in the cache.

We expect that the extra indexing action will add a constant overhead to the 
measurements of the first iteration, regardless of the number of entries.
Moreover, the latency times for the second iteration should remain the same, 
regardless of the number of entries.

\textbf{Note:} To get the most accurate results and since we want to test just 
the overhead of our indexing mechanism, cached has been configured to use one 
thread.  Moreover, the I/O depth is also one, to make sure that latency times 
are not skewed by the time the request waits in the request queue.

We present the results in Figure \ref{fig:cold1.pdf}.

\diagram{Latency performance of cold/warm cache per number of 
	objects}{cold1.pdf}

The focal point in these results is that the write latency, either of cold or 
warm caches, remains practically the same as the number of objects increases.  
This reinforces our statement that cached has average case O(1) complexity.

Finally, an interesting finding from our results is that the latency of 
indexing an entry, even in a hash table that supports 2 million objects (and 
potentially 8TBs of data), is on average less than 1μs, which is a very strong 
feature of our implementation.

\begin{comment}
As a side note, we observe a constant decrease in writelatency as the number of 
objects increase this is not something that should be attributed to our 
implementation. (explain that we have used a hash table that holds 2million 
objects, so it is not mapped to our process's address space. When more objects 
are indexed, the hash table becomes fuller and the latency of mmap()s is 
equally distributed to the objects. Else, the hash table is more scarce but the 
same blocks are hit, albeit not fully written, and thus the mmap latency is the 
same but distributed to less objects.)
\end{comment}

\section{Performance evaluation of a VM over Archipelago and cached}
\label{sec:vm-plot}

In this category, we will attempt to:
\begin{inparaenum}[i)]
	\item create a VM using Archipelago as its storage backend with cached 
		support	and
	\item use it to evaluate the real-life performance of various Archipelago 
		configurations, with and without cached.
\end{inparaenum}

The benchmarks of this section have been conducted on a vanilla Ubuntu 13.04 
VM. Once the VM was started successfully (which was an important feat in its 
own right), we installed FIO
\footnote{Flexible I/O tester, http://git.kernel.dk/?p=fio.git;a=summary}
, which is a fully-featured benchmark suite created by Jens Axboe, the 
maintainer of the Linux block layer.

Our aim was to simulate the workloads of the previous categories, in order to 
produce comparable results. The FIO job file that was created for this purpose 
is presented in Listing \ref{lst:job.ini}:

\plaintext{FIO job file}{job.ini}

The workload that this job file describes has been uniformly issued for every
Archipelago configuration that we have tested. We will illustrate this workload 
through an explanation of the job file parameters, which appear on the 
\textbf{[global]} section:

\begin{itemize}
	\item The request block size is \textbf{4KB} and is aligned on a 4KB 
		boundary,
	\item the total benchmark size is \textbf{1GB},
	\item the number of parallel requests is \textbf{16},
	\item the I/O requests bypass the VM's page cache, to ensure that the 
		results are not skewed due to guest caching.
\end{itemize}			

The following two sections, \textbf{[randwrite]} and \textbf{[randread]}, 
describe the two benchmark tests that were conducted. More specifically, in the 
first test we issue random writes that span the file and in the second test, we 
randomly read the written data.

Moreover, the fact that the benchmark is executed on a file means the following 
two things:

\begin{enumerate}
	\item That we are bound to include in our measurements the overhead of 
		filesystem operations and I/O scheduler of the VM. These overheads 
		however are essential for our benchmarks since they provide a more 
		accurate measurement of the I/O speed that the common user is likely to 
		experience.
	\item That we are guaranteed that only the first write to the file will be 
		affected by Copy-On-Write delays, while the subsequent writes will 
		behave normally.  Thus, we can discard the first write and use only
		the subsequent ones for our measurements.
\end{enumerate}

Finally, in addition to the bypass of the VM's page cache, we have also ordered 
the hypervisor to not use the page cache of the host. The only exception to 
this is the page cache test that evaluates the performance of the VM using the 
host's page cache in writeback mode.

We can now present the bandwidth and latency results for our benchmarks, which 
are illustrated in Figures \ref{fig:bw-vm.pdf} and \ref{fig:lat-vm.pdf} 
respectively. 

\diagram{Bandwidth performance of a VM over Archipelago}{bw-vm.pdf}

\diagram{Latency performance of a VM over Archipelago}{lat-vm.pdf}

We will explain which scenario each label represents:

\begin{description}
	\item[page-cache] \hfill \\
		This scenario tests solely the page cache of the \textbf{host} machine.  
		The rest of the scenarios bypass this page cache.
	\item[sosd] \hfill \\
		This scenario tests solely the performance of sosd peer.
	\item[cached-vast] \hfill \\
		This scenario uses cached with a large enough cache size, which means 
		that no evictions occur.
	\item[cached-unlimited] \hfill \\
		This scenario uses cached with half of the necessary space for 
		the benchmark but with the ability to cache practically 
		unlimited objects.
	\item[cached-limited] \hfill \\
		This scenario uses cached with half of the necessary space as 
		well as the ability to cache half of the necessary objects.
	\item[cached-writethrough] \hfill \\
		This scenario uses cached with half of the necessary space, 
		unlimited objects and in write-through mode.
\end{description}

		
Note that we have purposefully combined the write and read results on the same 
diagrams to make comparisons between them more evident. We will proceed with 
commenting on the performance results of each tested Archipelago configuration 
and we will invoke the previous synthetic benchmarks where necessary.

\subsubsection{Page-cache}

Our measurements clearly indicate that the page cache outperforms every 
Archipelago configuration. It also sets the mark for the upper speed limit that 
the Archipelago must strive to achieve.

Since page cache stores data in DRAM, we would expect an even better 
performance, given that our implementation manages to achieve latencies smaller
than 40μs, for the same number of parallel requests and block size, which can 
be seen in Figure \ref{fig:lat-write-comp-lie.pdf}.  We attribute this speed 
difference of < 1ms to the overhead of the paravirtualized storage, as well as 
the overhead of the VM operations (e.g. the filesystem, scheduler and kernel 
operations).

\subsubsection{Sosd}

The results for sosd are lower than the expected results. In fact, sosd has an 
additional latency of about 7ms for writes and 3ms for reads, which cannot be 
attributed to the < 1ms difference that we have observed in the page cache 
test. This means that besides the overhead that is incurred from the VM 
operations, there is also the latency of Archipelago, specifically of xsegbd, 
vlmcd and mapperd that has a performance penalty which is in the order of 
milliseconds.

\subsubsection{Cached-vast}

The comparison between sosd and this scenario shows that the performance 
speedup is approximately 4x for writes and 2x for reads. This is a very 
reinforcing result for our implementation and it proves that it does make a 
difference in an actual VM.

Arguably, the performance speedup that we measured in the synthetic benchmarks 
of previous categories was much bigger, yet we must keep in mind that the 
latency of the levels above cached do not allow this performance difference to 
show.  This statement is further reinforced by the comparison between these 
results and the results of the page-cache scenario; in both scenarios, data is 
cached in DRAM, which means that the main latency difference between them is 
the latency of Archipelago, which is approximately 2ms.

\subsubsection{Cached-unlimited}

The results of this scenario were expected, given the results of the previous 
scenario (cached-vast) and the synthetic benchmarks of Figures 
\ref{fig:lat-write-comp-truth.pdf} and \ref{fig:bw-write-comp-truth.pdf}.

For the first 50\% of the workload, the performance of cached-unlimited was the 
same as the performance of cached-vast. Predictably, it dropped after the 50\% 
mark but, interestingly, it was consistently better than the performance of 
sosd until the end, something that was not observed in the synthetic benchmarks 
we had conducted.  This is attributed to the fact that background flushes to 
sosd are supposed to take on average 3-4ms (see Figure 
\ref{fig:lat-write-comp-truth.pdf}), which is close to the time that a VM 
request needs to reach cached.  This means that evictions that happen in the 
background are essentially masked under the latency of the system.

We must also add that this scenario also achieves a \textbf{sustained} 2.5x 
performance speedup for  writes, while the peak behavior of this configuration 
is expected to be the same as cached-vast's. This shows that, even with a small 
cache, we can achieve a major performance improvement over sosd. 

Finally, we observe that the read performance is the same as sosd's but, as we 
have already explained, this happens due to background flushes that take place 
during the read operation, so the results are bound to be skewed.

\subsubsection{Cached-limited}

The revelation of our benchmarks has been this scenario, which is just as fast
as cached-vast for sustained writes. This is completely different from the 
results we observed in Figures \ref{fig:lat-write-comp-truth.pdf} and 
\ref{fig:lat-write-comp-truth.pdf}. It seems that the system latency completely 
masks the background flushes that take place, which is  probably due to the 
fact that evictions are more frequent but less batched.

\subsubsection{Cached-writethrough}

This scenario had predictable results. The write speed in writethrough mode was 
the same as the write speed of sosd whereas the read speed was a bit better, 
due to the fact that the data were cached. Overall, there was no particular 
performance improvement in writethrough mode.

