\chapter{Necessary theoretical background}\label{ch:theory}

In this typically boring section, we will explain the basic system facilities 
and programming concepts that are used by our implementation and Archipelago in 
general. More specifically, in Section ?...

\section{Interprocess Communication - IPC}

Interprocess Communication is a concept that predates the SMP
\footnote{Symmetric multiprocessing}
systems that we all use nowadays. It is a set of methods that an OS uses to 
allow processes and threads to communicate with each other. Archipelago for 
example, uses extensively IPC methods to synchronize its different components.

The full list of Linux's IPC methods is presented below:

\begin{itemize}
	\item \textbf{Signals:} they are sent to a process to notify it that an 
		event has occurred.
	\item \textbf{Pipes:} one-way channel that transfers information from one 
		process to the other.
	\item \textbf{Sockets:} bidirectional channels that can transfer 
		information between two or more processes either locally or remotely 
		through the network.
	\item \textbf{Message queues:} asynchronous communication protocol that is 
		used to exchange data packets between processes.
	\item \textbf{Semaphores:} Special purpose pipes that are used mainly for 
		process syncronization.
	\item \textbf{Shared memory:} a memory space that can be accessed and 
		edited by more than one process.
\end{itemize}


\section{Multi-threaded programming}

% http://accu.org/index.php/journals/1634

Multi-threading programming is good and is bad and here are some challenges:

\begin{enumerate}
	\item Concurrency control
	\item Challenge 2
	\item Challenge 3
\end{enumerate}

\subsubsection{Concurrency control}

\paragraph{Locking}

% Taken from wikipedia

Three concepts for locking:

\begin{enumerate}
	\item Lock overhead
	\item Lock contention
	\item Deadlocking
\end{enumerate}
