\begin{frame}[t,plain]
\titlepage
	\note[item]{Καλημέρα σας, ονομάζομαι Αλέξιος Πυργιώτης\\
		Θα σας παρουσιάσω τη διπλωματική μου με τίτλο:..}
	\note[item]{Ακούγεται κάπως περίεργο στα ελληνικά...\\
	αυτό που πραγματεύται είναι την δημιουργία ενός caching μηχανισμού για 
	το Archipelago, ένα distibuted, storage layer}
	\note[item]{Συγκεκριμένα, στην παρουσίαση αυτή θα μιλήσουμε για τον 
		cached, δηλαδή τον caching μηχανισμό μας, αλλά και για το 
		synapsed, ένα συμπληρωματικό εργαλείο που στόχος του είναι να 
		δώσει στον cached δικτυακές δυνατότητες}
\end{frame}

\section{Introduction}

\begin{frame}[t]{Contents}

	\begin{itemize}
		\item Introduction
		\item Request handling
	\end{itemize}

	\note[item]{O κορμός της παρουσίασης είναι ο εξής:
		\begin{itemize}
			\item Αρχικά, παρουσιάζουμε κάποια εισαγωγικά που 
				αφορούν το background της εργασίας μας.  
				Αναφέρουμε τι είναι το Synnefo, τι είναι η 
				υπηρεσια ~okeanos και τι είναι το Αρχιπέλαγο
			\item Έπειτα, δείχνουμε τον τρόπο με τον οποίο η 
				υποδομή μας χειριζεται αιτήματα δεδομένων από 
				ένα VM.
		\end{itemize}
	}

\end{frame}

\begin{frame}{Synnefo}
	\note{Ας ξεκινήσουμε με την παρούσα κατάσταση.\\
	Το software που τα ξεκίνησε όλα είναι το Synnefo}

	\includegraphics{images/synnefo-logo.png}

	Open source, production-ready, cloud software.\\
	Designed since 2010 by GRNET.
	\note{\dspc..by GRNET -> Και φυσικά τα παιδιά που βλεπετε εδώ}
	
	\spc
	Synnefo, as most cloud software, has the following services:
	\begin{itemize}
		\item Compute Service
		\item Network Service
		\item Storage Service
		\item Image Service
		\item Identity Service
	\end{itemize}

	\note{\dspc
		\begin{itemize}
			\item Compute service, είναι η υπηρεσία η οποία 
				προμηθεύει τους χρήστες με VMs και επιτρέπει το 
				χειρισμό τους
			\item Network service, είναι η υπηρεσία η οποία δίνει 
				τη δυνατότητα στους χρήστες να δημιουργήσουν 
				ιδιωτικά δίκτυα και να συνδέσουν τα VMs τους σε 
				αυτά.
			\item Storage service, που κοινώς αποθηκεύει τα αρχεία 
				των χρηστών. Στην περίπτωση του Synnefo όμως, 
				έχουμε ένα κοινό σημείο για τα πάντα: είτε 
				είναι αρχεία, είτε δίσκοι των VMS, είτε images 
			\item Image Service, υπεύθυνο για το deployment ενός VM 
				από ένα image. Επίσης, κάνει και 
				παραμετροποιήσεις (παράδειγμα ssh κλειδιά)
		\end{itemize}
		image service είναι\\
	}
\end{frame}

\begin{frame}{okeanos}

	\includegraphics{images/okeanos-logo.png}

	\begin{itemize}
		\item IaaS service
		\item Targeted at the Greek Academic and Research Community
		\item Designed by GRNET
		\item In production since 2011
	\pause
		\item ...and of course powered by Synnefo.
	\end{itemize}

	\note{
		\begin{itemize}
			\item IaaS είναι πρακτικά η παροχή εικονικής υποδομής 
				σε χρήστες (δηλαδή πάρε υπολογιστή (VM), 
				δίκτυα, αρχεία κτλ)
			\item Δωρεάν για τους Ακαδημαϊκους σκοπούς, ήδη 
				γίνονται εργαστήρια στο EMP και απ' αυτό το 
				εξάμηνο σε άλλες σχολές
		\end{itemize}
	}
				
\end{frame}


