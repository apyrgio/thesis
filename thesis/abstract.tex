\begin{abstractgr}
	Η υπηρεσία αποθήκευσης δεδομένων μιας cloud υποδομής έχει κατά κανόνα 
	υψηλές απαιτήσεις σε επιδόσεις, ειδικά όταν είναι επιφορτισμένη με την 
	παροχή των εικονικών δίσκων για τις Εικονικές Μηχανές (VMs) των χρηστών. Σε 
	μεγάλες εγκαταστάσεις όμως, παρατηρείται ότι η επίδοση των σκληρών δίσκων 
	(hard disks) συνήθως δεν επαρκεί για την εξασφάλιση γρήγορων ταχυτήτων για 
	τις ανάγκες των χρηστών. Αυτό οδηγεί τους προγραμματιστές στο να 
	καταφεύγουν σε λύσεις προσωρινής αποθήκευσης δεδομένων (caching) σε 
	γρηγορότερα αποθηκευτικά μέσα, για να βελτιώσουν τις επιδόσεις τους.
	
	Η παρούσα διπλωματική εργασία παρουσιάζει τον \textit{cached}, έναν 
	caching μηχανισμό ο οποίος χρησιμοποιεί ως αποθηκευτικό μέσο τη φυσική 
	μνήμη (RAM) και δημιουργήθηκε για τη βελτίωση της επίδοσης του 
	Archipelago, ενός κατανεμημένου αποθηκευτικού συστήματος.	
	Επιπροσθέτως, παρουσιάζεται το συμπληρωματικό δικτυακό εργαλείο 
	\textit{synapsed}, το οποίο ανοίγει το δρόμο για την αναβάθμιση του 
	\textit{cached} σε έναν πλήρως κατανεμημένο caching μηχανισμό.
	
	Τα αποτελέσματα από τις πρώτες μετρήσεις είναι αρκετά ενθαρρυντικά και 
	δείχνουν ότι η χρήση του \textit{cached} μπορεί να αυξήσει την επίδοση του 
	Archipelago μέχρι και 400\%.
	\begin{keywordsgr}
		cached, synapsed, synnefo, okeanos, rados, archipelago, cloud, εικονική 
		μηχανή, προσωρινή αποθηκευτική μνήμη, Ο(1) πολυπλοκότητα, φυσική μνήμη, 
		κατανεμημένο σύστημα
		\end{keywordsgr}
\end{abstractgr}

\begin{abstracten}
	The performance of the storage service of a cloud infrastructure is very 
	critical, especially when it provides virtual volumes for the Virtual
	Machines of users. Underlying hard disks however, are usually inadequate in 
	providing fast performance for large deployments and the storage engineers 
	commonly have to resort in using various buffering and caching techniques.
	
	This diploma thesis presents \textit{cached}, an in-memory caching 
	mechanism that aims to improve the performance of Archipelago, a 
	distributed storage layer. Additionally, this diploma thesis introduces 
	\textit{synapsed}, a complementary network tool that paves the way for 
	the creation of a fully distributed cache.
	
	Early performance evaluations look very promising and show that 
	\textit{cached} can improve the current Archipelago performance up to 
	400\%.
	\begin{keywordsen}
		cached, synapsed, synnefo, okeanos, rados, archipelago, cloud, 
		virtual machine, volume service, cache, hash table,  O(1) 
		complexity, ram, distributed system, replication
	\end{keywordsen}
\end{abstracten}

\begin{acknowledgementsgr}
	\todo
\end{acknowledgementsgr}
